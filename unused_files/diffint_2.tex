\section{Introduction}

We have seen how a beam of light behaves as it passes through different 
media. Since Maxwell's electrodynamics indicates that light is really a 
wave, the principle approximation that allowed us to treat these waves as
a beam involved the smallness of the wavelength of light. Since all the 
objects that we put in the way of the light were much, much bigger than
an individual wavelength, the ray approximation worked quite well. Here,
we will use objects whose dimensions are approximately those of the 
wavelength of light. These will then reveal the wave-like properties of 
these waves.

The principle difference between rays and waves lies in the 
superposition principle. This leads to interference phenomena, and
specifically here, to diffraction effects for waves. We will observe several 
different types of diffraction patterns, due to one, several and many
individual slits or openings. We will verify some qualitative predictions
of the superposition principle and use these effects to measure the 
wavelength of the laser light source. These ideas form the basis for some
of the most precise measurements ever made; furthermore, these notions 
underlie most of the technology used in manufacturing computer chips.

\section{Theory}

\subsection{References}

Serway addresses the interference of light waves in Chapter~37 
(Interference of Light Waves), and cursorily discusses diffraction
in Chapter~38 (Diffraction and Polarization). The single slit
diffraction pattern appears in Section~38.2, and the diffraction
grating shows up in Section~38.4. We will discuss these and other
effects here.

\subsection{Mathematical Preliminaries}

In order to give a coherent and efficient treatment of the interference 
of many waves, we need to introduce some formulas that you may not have 
seen before. The first bit is called {\em Euler's formula}:
\begin{eqnarray}
e^{ix} = \cos x + i \sin x, \label{eq:diff:euler}
\end{eqnarray}
where $i = \sqrt{-1}$ is the complex unit. This relationship between the 
complex exponential and trig functions will help us streamline our 
calculations. To derive~(\ref{eq:diff:euler}), examine the Taylor series 
for the exponential:
\begin{eqnarray*}
e^{ix} = \sum^\infty_{n=0} \frac{(ix)^n}{n!};
\end{eqnarray*}
We can now use the periodicity of complex multiplication to separate this
series into even and odd parts:
\begin{eqnarray*}
e^{ix} & = & \sum^\infty_{n=0} \frac{(ix)^n}{n!},\\
       & = & \sum^\infty_{n=0} \frac{(ix)^{2n}}{(2n)!} +
             \sum^\infty_{n=0} \frac{(ix)^{2n+1}}{(2n+1)!},\\
       & = & \sum^\infty_{n=0} \frac{(-1)^n x^{2n}}{(2n)!} +
            i\sum^\infty_{n=0} \frac{(-1)^n x^{2n+1}}{(2n+1)!},\\
       & = & \cos x + i \sin x,
\end{eqnarray*}
where to move from the second to the third line we used $i^{2n} = (-1)^n$ 
and $i^{2n+1} = i(-1)^n$ and then we identified the remaining series as
the sine and cosine series.

The second bit of mathematical trickery is the geometric series:
\begin{eqnarray*}
\sum^N_{n=0} x^n = \frac{1-x^{N+1}}{1-x}
\end{eqnarray*}
To see how to derive this, write out the left-hand side:
\begin{eqnarray*}
\sum^N_{n=0} x^n = 1 + x + x^2 + \cdots + x^N
\end{eqnarray*}
Now, rewrite $1$ as $\frac{1-x}{1-x}$ and multiply this sum by this form 
of $1$: 
\begin{eqnarray*}
\sum^N_{n=0} x^n & = & (1 + x + x^2 + \cdots + x^N) \frac{1-x}{1-x},\\
    & = & ( 1 + x + x^2 + \cdots + x^N\\ 
    &   & \phantom{(1} - x - x^2 - \cdots - x^N - x^{N+1} )
                   \frac{1}{1-x},\\
    & = & \frac{1-x^{N+1}}{1-x}.
\end{eqnarray*}

For most of the experiments we will do, the angles involved will be quite 
small. There are some handy approximations we can use in this case to simplify
our trigonometric calculations and assist us in extracting qualitative 
behavior. Recall that the leading term in the Taylor series expansion of 
$\sin \theta$ is simply $\theta$; so if $\theta$ is small, the higher order 
terms are unimportant. So,
\begin{eqnarray*}
\sin \theta \sim \theta.
\end{eqnarray*}
Now, if we have a right triangle with $\theta$ as one of the acute angles, and
legs with lengths $y$ and $L$, $L \gg y$ (see Figure~\ref{fig:diff:triangle}), 
then we can approximate
\begin{eqnarray*}
\sin \theta &\sim& \theta \\
\frac{y}{\sqrt{y^2+L^2}} &=& \frac{y}{L}
\frac{1}{\sqrt{1+\frac{y^2}{L^2}}}
\sim \frac{y}{L}
\end{eqnarray*}
so that we arrive at the {\em small angle approximation} $\theta \sim y/L$. We
will make extensive use of this relation.
Now we can move on to the more interesting physics applications of these 
relations.
\begin{figure}[htb]
\centerline{\epsfxsize=6cm \epsfbox{diffint_2/triangle.eps}}
\caption{The triangle used to relate $\sin\theta$ to $y$ and $L$.} 
\label{fig:diff:triangle}
\end{figure}

\subsection{Interference of Monochromatic Waves}

Interference occurs when two waves overlap. For water waves, the net 
displacement of the surface at any given point is the sum of the displacements
due to the component waves. In this case of light, the electric fields and 
magnetic fields each add together via the superposition principle. 
We will only consider the rather ideal case of the interference of two waves 
that have the same frequency; such waves go by the name {\em monochromatic} 
waves.

To get things rolling, consider a monochromatic light wave that occupies some 
region of space. We can represent its electric field magnitude in the form
\begin{eqnarray*}
E = E_0 \cos (\omega t - k x + \phi)
\end{eqnarray*}
where $E_0$ is the maximum amplitude of the wave, $(x,t)$ is the location and 
time at which we measure the wave's amplitude, $\omega$ is the angular 
frequency of the wave, $k$ is the wave number $2 \pi/\lambda$ with $\lambda$
being the wavelength and $\phi$ is the phase of the wave. 
Using~(\ref{eq:diff:euler}), we can write this as
$$ E = E_0 {\cal R}e\left[e^{i(\omega t - k x + \phi)}\right],$$
where ${\cal R}e$ stands for taking just the real part of the complex 
expression. Now, suppose we have two such monochromatic waves with the same
frequency and wavelength, each with its own phase. Then, we can write
\begin{eqnarray*}
E_1 = E_{0} e^{i(\omega t - k x + \phi_1)},\\
E_2 = E_{0} e^{i(\omega t - k x + \phi_2)}.
\end{eqnarray*}
Since these two waves occupy the same region of space, they will interfere;
the resultant amplitude of the combined wave will then be
\begin{eqnarray*}
E_1+E_2 & = & E_0 \left( e^{i(\omega t - k x + \phi_1)} +
                         e^{i(\omega t - k x + \phi_2)} \right)\\
        & = & E_0 e^{i(\omega t - k x)} \left(e^{i\phi_1}+e^{i\phi_2}\right)\\ 
        & = & E_0 e^{i(\omega t - k x)} \left( 
               e^{i\phi_1/2}e^{i\phi_1/2}e^{i\phi_2/2}e^{-i\phi_2/2}
            +  e^{i\phi_2/2}e^{i\phi_2/2}e^{i\phi_1/2}e^{-i\phi_1/2} \right)\\
        & = & E_0 e^{i(\omega t - k x)} e^{i(\phi_1+\phi_2)/2} 
            \left( e^{i(\phi_1-\phi_2)/2} + e^{-i(\phi_1-\phi_2)/2} \right)\\
        & = & 2E_0 \cos \left( \frac{\phi_1-\phi_2}{2} \right) 
                e^{i(\omega t - k x + (\phi_1+\phi_2)/2)}
\end{eqnarray*}
where we've used the fact that $\cos x = \frac{1}{2} (e^{ix}+e^{-ix})$. This 
expression says some interesting things about the resultant amplitude which 
consists of everything not in the exponential. If $(\phi_1-\phi_2)/2 = 
(2n+1) \frac{\pi}{2}$ for some integer $n$, then the amplitude is $0$! This is 
the condition for destructive interference. Similarly, if $(\phi_1-\phi_2)/2
= n \pi$ then the amplitude takes on its extreme values of $\pm 2E_0$ 
according to whether $n$ is even or odd; this is the condition for constructive
interference. Thus, we see that the resultant amplitude depends critically on
the phase difference $\phi_1-\phi_2$ of the two waves; this is what we will 
need to calculate for specific setups.

\subsection{Two Slit Interference}

Consider the diagram in Figure~\ref{fig:diff:2-slit}. It shows two openings of 
width $a$ whose centers are separated by a distance $d$. We want to calculate 
the intensity as a function of position on a screen a distance $L$ away. 
Consider point~P which makes an angle of $\theta$ with respect to the center 
normal. If we illuminate both slits from the left with a monochromatic wave, 
we can consider each slit as a source of light; to compute the intensity we 
need to determine the phase difference between the waves coming from each slit.
Then our two wave interference analysis will tell us the amplitude, from which 
we can get the intensity.
\begin{figure}[htb]
\centerline{\epsfxsize=10cm \epsfbox{diffint_2/twoslit.eps}}
\caption{The geometry of a simple two slit interference experiment.}
\label{fig:diff:2-slit}
\end{figure}

If the screen is quite far away from the slits, $d \ll L$, then the two rays
between P and each slit are practically parallel. The phase difference arises
from the additional distance one ray must travel over the other. From the 
geometry of Figure~\ref{fig:diff:2-slit}, we see that this distance is
$ \ell = d \sin \theta.$ Thus, the angular phase shift that arises from this 
distance is then $2 \pi \ell/\lambda = k \ell$. Thus, the intensity
maxima should occur when
\begin{eqnarray*}
\frac{k \ell}{2} = n \pi;
\end{eqnarray*}
that is, when
\begin{eqnarray}
\sin \theta = n \frac{d}{\lambda}.~~~{\rm (maxima)}
\label{eq:diff:2 slit maxima}
\end{eqnarray}
In the small angle approximation this becomes
\begin{eqnarray*}
\fbox{$ \displaystyle y_n=n \lambda \frac{L}{d} $}
\end{eqnarray*}
where $y_n$ is the distance from the centerline to the $n$th maximum.
We can also predict where the minimum intensity should occur; simply set
$k \ell = (2n+1) \frac{\pi}{2}$, which becomes
\begin{eqnarray}
\sin \theta = \frac{2n+1}{2} \frac{d}{\lambda}. ~~~{\rm (minima)}
\label{eq:diff:2 slit minima}
\end{eqnarray}
which has the small angle form
\begin{eqnarray*}
y_n = \frac{2n+1}{2} \lambda \frac{L}{d}.
\end{eqnarray*}
We can gain some quick intuition about the behavior of this interference 
pattern back looking at the widths of the primary maxima; this quantity, $w$, 
we define as the distance between successive minima:
\begin{eqnarray*}
w = y_{n+1}-y_n \sim \lambda \frac{L}{d}
\end{eqnarray*}
which indicates that the width grows as the distance between the slits 
decreases. This is a common feature of interference and diffraction, as we will
see.


\subsection{Multiple Slit Interference}

Now, we consider what happens when we have a situation with more than 2 slits, 
say $n$ slits. We will see that our complex formalism, which was quite awkward 
in the 2 slit case, will serve us well. Consider the configuration in 
Figure~\ref{fig:diff:n-slits}. 
\begin{figure}[htb]
\centerline{\epsfxsize=10cm \epsfbox{diffint_2/multislit.eps}}
\caption{Multiple slit geometry.}
\label{fig:diff:n-slits}
\end{figure}
At point P on the screen, there will now be 
$n$~waves interfering. We need to determine all the relative phase shifts and 
add them up. Keeping things reasonable, let's consider the situation for which 
all the slits have the same width $a$ and separation $d$ and all are 
illuminated by the same monochromatic wave from the left. Labeling the rays 
with the one that must travel the farthest first, we need to perform the sum
\begin{eqnarray*}
\sum^n_{j=1} E_j & = & \sum^n_{j=1} E_0 e^{i(\omega t - k x + \phi_j)}\\
                   & = & E_0 e^{i(\omega t - k x)} \sum^n_{j=1} e^{i\phi_j}
\end{eqnarray*}
for the total electric field at P. $\phi_j$ is the phase difference of the 
$j$-th ray from the first. From the geometry, we see that the phase shift 
between each successive ray is $\phi_0 = k \ell = -kd \sin \theta$. Thus, the 
difference between the first and $j$th rays is $\phi_j=(j-1)\phi_0$. Plugging 
this into our sum, we find
\begin{eqnarray*}
\sum^n_{j=1} E_j & = & E_0 e^{i(\omega t - k x)} \sum^{n-1}_{j=0} 
\left( e^{i\phi_0} \right)^j\\
& = & E_0 e^{i(\omega t - k x)} \frac{1-e^{in\phi_0}}{1-e^{i\phi_0}}\\
& = & E_0 e^{i(\omega t - k x)} e^{i\frac{(n-1)\phi_0}{2}}
      \frac{e^{in\phi_0/2}-e^{-in\phi_0/2}}{e^{i\phi_0/2}-e^{-i\phi_0/2}}\\
& = & E_0 e^{i\left(\omega t - k x + \frac{(n-1)\phi_0}{2}\right)}
       \frac{\sin (n\phi_0/2)}{\sin (\phi_0/2)}
\end{eqnarray*}
where we have used the fact that $\sin x = \frac{1}{2i}(e^{ix}-e^{-ix})$ to 
obtain the last line. So, the real electric field is the real part of this 
expression:
\begin{eqnarray*}
E = E_0 \frac{\sin (n\phi_0/2)}{\sin (\phi_0/2)}
        \cos\left(\omega t - k x + \frac{(n-1)\phi_0}{2}\right), 
\end{eqnarray*}
from which we obtain the time averaged intensity as
\begin{eqnarray*}
I & = & I_0 \frac{\sin^2 (n\phi_0/2)}{\sin^2 (\phi_0/2)},\\
  & = & I_0 \frac{\sin^2 \left(n \pi \frac{d}{\lambda} \sin \theta\right)}
                 {\sin^2 \left(  \pi \frac{d}{\lambda} \sin \theta\right)}
\end{eqnarray*}
where $I_0 = \epsilon_0 E_0^2/2$. As an exercise you can show that, when $n=2$,
this takes the form
$$I=4I_0\cos^2\left(\pi \frac{d}{\lambda} \sin \theta\right).$$
This two slit intensity is plotted as a function of the angle~$\theta$ in in 
Figure~\ref{fig:diff:2slitint}.  Figure~\ref{fig:diff:n slit intensity} 
illustrates the intensity pattern for 4~slits.
\vfill
\begin{figure}[htb]
\centerline{\epsfxsize=10cm \epsfbox{diffint_2/2slitint.eps}}
\caption{Two slit ($n=2$) intensity pattern for $\lambda=10d$.}
\label{fig:diff:2slitint}
\end{figure}
\begin{figure}[htb]
\centerline{\epsfxsize=10cm \epsfbox{diffint_2/multiint.eps}}
\caption{Multiple slit intensity pattern for $n=4$ and $\lambda=10d$.}
\label{fig:diff:n slit intensity}
\end{figure}

\subsection{Single Slit Diffraction}
\label{sec:diff:singleslit}

If we place a single slit in front a monochromatic beam, we find a surprising
result: the beam gets distorted and fringes appear in the shadow zone of slit.
To describe this phenomenon, known as diffraction, we can use Huygen's 
principle. According to this idea, we can treat the single slit as a bunch of
really, really small single slits, and let them interfere with one another. So,
let our real slit have a width $a$ and consider a small subset of this region
with a width $\Delta y$ located a distance $y$ from the center, as shown in 
Figure~\ref{fig:diff:single slit diffraction}.
\begin{figure}[htb]
\centerline{\epsfxsize=10cm \epsfbox{diffint_2/singlediff.eps}}
\caption{Configuration for analyzing single slit diffraction using Huygen's
principle.}
\label{fig:diff:single slit diffraction}
\end{figure}
Now, the phase difference between a ray at  our minislit at $y$ and one at
the center is $\phi(y)  = k y \sin \theta$, where we have again 
assumed that the rays leaving are practically parallel. Thus, if our 
observation point P is a distance $x$ from the slit's center, we can write
the electric field field at P due to the minislit at $y$ as
\begin{eqnarray*}
E(y) = E_0(y) e^{i(\omega t - k x + \phi(y))}.
\end{eqnarray*}
The initial electric field will only be a small fraction of the electric field 
passing through the entire slit; you can think about this in terms of energy 
flux. The amount of energy passing through the minislit of width $\Delta y$, 
is $\left( \Delta y/a \right)^2$ times the amount of energy passing through 
the entire slit, since energy flux depends on the area. So, the amount of 
field at this slit is $\Delta y / a$ of the field for the entire slit, $E_0$. 
So, $E_0(y) = (\Delta y/a) E_0$; then
\begin{eqnarray*}
E(y) = \frac{E_0}{a} \Delta y e^{i(\omega t - k x + k y \sin \theta)}.
\end{eqnarray*}
Adding up all the contributions of all these slits and taking the minislit
widths to zero yields the following integral
\begin{eqnarray*}
E & = & \frac{E_0}{a} e^{i(\omega t - k x)} \int^{a/2}_{-a/2} e^{iky \sin 
\theta)} dy\\
& = & E_0 e^{i(\omega t - k x)} \left(  \frac{\sin ( \frac{ka}{2}
\sin \theta )}{\frac{ka}{2} \sin \theta} \right).
\end{eqnarray*}
Notice the similarity between this calculation using the integral and the 
finite multiple slit result using the geometric series. It provides an almost 
identical result. Taking the real part, squaring it and averaging over time 
leads to an analogous expression for the intensity in this case. A plot of this
function appears in Figure~\ref{fig:diff:single slit diffraction intensity}.\\
\begin{figure}[htb]
\centerline{\epsfxsize=11cm \epsfbox{diffint_2/singslitpatt.eps}}
\caption{Intensity pattern for a single slit diffraction experiment.}
\label{fig:diff:single slit diffraction intensity}
\end{figure}

\noindent Within the small angle approximation, the positions of the minima 
are 
\begin{eqnarray*}
y_n = n \lambda \frac{L}{a}
\end{eqnarray*}
which is the same as the two slit interference result with the slit separation
$d$ replaced by the slit width $a$. This yields a fringe width $w$ proportional
to $\lambda L/a$, showing that the fringes expand as the slit width decreases. 
\suppressfloats

\subsection{Multiple Slit Diffraction}
\label{sec:diff:multislit}

You can go through a similar calculation for multiple slit diffraction 
patterns. But, we can also be a bit smarter about it and save ourselves some 
work. Notice that the diffraction of a single slit will not affect the 
treatment of interference of multiple slits because you simply add the 
corresponding contributions from the various slits, with the appropriate phase
factors. Because of the exponential dependence on this phase, the result simply
becomes the product of the single slit diffraction pattern and the multiple 
slit interference pattern. 
\begin{eqnarray}
I = I_0 \frac{\sin^2 \left(\frac{n\pi d}{\lambda} \sin \theta \right)}{
\sin^2 \left(\frac{\pi d}{\lambda} \sin \theta\right)}
        \frac{\sin^2 \left( \frac{\pi a}{\lambda} \sin \theta\right)}{
        \left( \frac{\pi a}{\lambda} \sin \theta  \right)^2}
    \label{eq:diff:multislit diffraction intensity}
\end{eqnarray}
This is what you will observe in the lab. We provide a graph of this intensity 
in Figure~\ref{fig:diff:multislit diffraction} for $n=2$.
\begin{figure}
\centerline{\epsfysize=8cm \epsfbox{diffint_2/envelope.eps}}
\caption{Multislit diffraction pattern for $n=2$.}
\label{fig:diff:multislit diffraction}
\end{figure}

From this relation, we can find out where the intensity minima are. To 
accomplish this, we must make the numerators zero while keeping the 
denominators nonzero. For the interference part, this condition yields
\begin{eqnarray*}
n \pi \frac{d}{\lambda} \sin \theta = m \pi, \hspace{2cm} m \neq 0
\end{eqnarray*}
where the condition on the integer $m$ arises from keeping the denominator
finite. Thus, the angles for a minimum intensity satisfy
\begin{eqnarray*}
\sin \theta = \frac{m}{n} \frac{d}{\lambda}
\end{eqnarray*}
for $m$ and $n$ incommensurate ($m\neq n$). Now, for the diffraction part, we 
don't need to worry about the denominator vanishing except at $\theta=0$; this 
leads us to the minimum condition
\begin{eqnarray*}
\sin \theta = m \frac{\lambda}{a}, \hspace{2cm} m \neq 0.
\end{eqnarray*}
Since $a < d$, the diffraction minima occur farther apart than the 
interference minima; this means that the diffraction really forms an envelope
of the interference patterns. The maxima are harder to locate, since they 
involve the interplay between the denominator and the numerator. To find them,
you must take the derivative and set it equal to zero. This will yield an
unpleasant transcendental equation, that will tell us that the maxima are 
roughly half way in between the minima, but a little closer to the center.

\subsection{Diffraction Grating}

We can use our multislit diffraction pattern to examine a very useful device:
the diffraction grating. This is a plate that has thousand of slits cut into 
it. For this large $n$ limit, we can easily locate the intensity maxima. 
Consider equation~(\ref{eq:diff:multislit diffraction intensity}) with $n$ very
large. The interference part produces sharp peaks of very high intensity, 
scaling as $n^2$, separated by many ($n-1$) practically invisible secondary
maxima. The diffraction part simply provides an overall decrease in the
intensity of these peaks away from the center. So, we can focus our attention
on the interference part to extract where the primary peaks are.

So, looking at the interference part of 
equation~(\ref{eq:diff:multislit diffraction intensity}), to make $I$ a 
maximum, we need the arguments of both the top and the bottom to go to zero 
simultaneously. If this is the case, then L'H\^opital's rule gives the value 
$n^2$ for the ratio. This occurs when
\begin{eqnarray*}
\pi \frac{d}{\lambda} \sin \theta = m \pi
\end{eqnarray*}
for some integer $m$. Rearranging this condition gives the {\em grating
equation}:
\begin{eqnarray*}
\fbox{$ \displaystyle d \sin \theta = m \lambda; $}
\end{eqnarray*}
$m$ is called the {\em order} of the fringe. Knowing the slit spacing~$d$, the 
order of the fringe and measuring $\sin \theta$ will produce a fairly precise
measurement of the wavelength.

\section{Apparatus}

The apparatus for this lab is rather simple.  We'll use the lasers as light 
sources, since we need a monochromatic beam of light to clearly illustrate the
effects of interference and diffraction. Please recall the guidelines for 
laser safety given in Lab~\ref{ch:optics}. We'll use an adjustable single
slit to study single slit diffraction, a slide with multislit patterns, and
a diffraction grating slide.

\vfill
\pagebreak

%  Label worksheets by \thechapter.W
\renewcommand{\thesection}{\thechapter.W}

\section{Diffraction and Interference Worksheet}
{\bf \Large Name:}~ \rule{5cm}{.1mm}~~~~~~~
{\bf \Large Day/Time:}~\rule{3cm}{.1mm}\\ 
\ \\
{\bf \large Partners' Names:}~\rule{6cm}{.1mm}\\

\subsection{In-Lab Procedure}

For most of the procedures in this lab, we will require a distance of at least
2~m between the slit assembly and the screen you'll use to observe the pattern;
$L$ must be large for the small angle approximation.  As a screen, we'll 
use large sheets of computer paper taped to the wall; these are nice, because 
you can easily trace the pattern on them.  However, this means that several 
groups of students will have to aim the lasers {\it across} the room.  As long
as everyone is careful not to aim their own lasers up at eye level and we 
remain aware of everyone else's lasers, there should be no problems.  Be 
patient and understanding if your instructor or a classmate needs to walk in
front of your beam when you're in the middle of making a sketch.
 
\subsubsection{Single Slit Diffraction}  

Find a convenient way to aim the laser at one of the walls so that there's
no electrical outlet or other obstruction to taping your paper screen onto the
wall; make sure that you will have at least a 2~m separation between the 
{\it slit assembly} and the wall.  Place the single slit apparatus in front of 
the laser and align the slit with the beam. This may take a bit of playing 
around; try placing the slit apparatus on its side or prop it up if necessary. 
You might also need to adjust the slit width; just be careful not to look 
directly into the laser beam or its reflections.  When you have the slit set 
up properly, you should see a pattern of {\it spots}. \\ 
\vspace*{.3cm} \\
{\bf Question 0}: What does this pattern, 
specifically the brightness of the spots, have to do with 
Figure~\ref{fig:diff:single slit diffraction intensity}?   \\

\clearpage

\noindent Trace {\bf two} patterns: one with a relatively large slit 
width and another with a 
relatively small slit width; make sure that you label these appropriately.   
You will use these traces to answer a series of questions in the In-Classroom
Calculation \& Analysis section.  


\subsubsection{Multiple Slit Diffraction/Interference Patterns} 
\label{sec:diff:multislitproc}

\noindent Using a length of string and a meterstick, 
measure distance from slit 
to screen, $L$ and record this value below with uncertainty.
\begin{center}
$L=$~ \rule{3cm}{.1mm} 
\end{center}
Replace the single slit assembly with the multi-slit slide (mounted in a 
magnetic holder).  Adjust the slide
so that the two slit pattern (slit pattern, not interference pattern) is 
directly in front of the beam; adjust it until you get a clear interference 
pattern on the screen. Record the slit width $a$ and the slit spacing 
$d$ below.

\begin{center}
$a=$~ \rule{3cm}{.1mm}~~~~ $d=$~ \rule{3cm}{.1mm}
\end{center}
\vspace*{.5cm}

\begin{figure}[htb]
\centerline{\epsfxsize=15cm \epsfbox{diffint_2/interpatt.eps}}
\caption{The interference/diffraction pattern for 5 slits.}
\label{fig:diff:interpatt}
\end{figure}

\noindent Trace the {\bf two} slit pattern (remember to label it). 
When doing the trace,
make sure you are able to trace at least {\bf eight primary} maxima. (Four
primary maxima on each side of the central maximum is preferable).
Determine the orders of each of the primary maxima on your trace of
the two-slit interference pattern. Refer to Figure~\ref{fig:diff:interpatt}
for guidance but note that this figure is of a 5-slit pattern.
Measure the distances $y_n$ for at least 8 primary maxima; remember to skip
the secondary maxima and missing orders when you do this; once again 
referring  to Figure~\ref{fig:diff:interpatt} for guidance. \\

\noindent
Measure the distances $y_n$ from the $n$th primary maximum to the
center of the central maximum for at least eight primary maxima. {\bf
Note:} The values of $y_n$ for negative $n$ should be taken to be
negative. Record your eight $y_n$ vs. $n$ measurements into 
Table~\ref{tab:DI:twoslit}.

\begin{table}[htb]
\begin{center}
\begin{tabular}{|c|c|c|c|c|}
\hline
\multicolumn{5}{|c|}{Measurements of $y_n$ vs. $n$ for Two-slit Diffraction Pattern.} \\
\hline
Distance ($y_n$) & Order ($n$) & & Distance ($y_n$) & Order ($n$) \\
\hline
\hspace*{3cm} & \hspace*{3cm} & \hspace*{.3cm} & \hspace*{3cm} & \hspace*{3cm} \\
& & & & \\
\hline
& & & & \\ & & & & \\
\hline
& & & & \\ & & & & \\
\hline
& & & & \\ & & & & \\
\hline
\end{tabular}
\end{center}
\caption{$y_n$ vs. $n$ measurements for two-slit diffraction pattern.}
\label {tab:DI:twoslit}
\end{table}

\ \\
\noindent 
Let's also examine the qualitative properties of the other slit
patterns.  Observe the patterns for the 3, 4, and 5 slit patterns; you
don't need to sketch them.  Is there anything discussed in
Section~\ref{sec:diff:multislit} that you can compare this behavior
with?  Record below the results of your observations of the {\it
qualitative} properties of the interference patterns resulting from 3,
4, and 5 slits. In particular, answer the following two questions. 
\vspace*{.5cm}

\noindent 
{\bf Question 1}: For each of the 3, 4, and 5 slit patterns, how many
secondary maxima appear between primary maxima?
\vspace*{.8cm}

\noindent
{\bf Question 2}: What happens to the size of the primary maxima as you
increase the number of slits?\\








\subsubsection{Diffraction Grating Pattern} 

Replace the multi-slit slide with the diffraction grating slide.  Can you see
the diffraction pattern?  What if you place a sheet of paper directly in front
of the grating?  Move the laser and grating so that you decrease the distance
between the grating and the wall. Adjust the grating-to-screen distance until 
you can see five dots, as illustrated in Figure~\ref{fig:diff:diffgratpat}. 
\begin{figure}[htb]
\centerline{\epsfxsize=8cm \epsfbox{diffint_2/diffpatt.eps}}
\caption{The interference pattern from the diffraction grating.} 
\label{fig:diff:diffgratpat}
\end{figure}

\noindent
Measure the grating-to-screen distance, $L_g$, and write down the number of 
lines/cm, $s,$
as printed on the grating slide. 

\begin{center}
$L_g=$~ \rule{3cm}{.1mm} ~~~~ $s=$~ \rule{3cm}{.1mm}
\end{center}
\vspace*{.5cm}
\noindent  Now trace the 5~bright spots that form the 
interference pattern. Some dim spots may appear above and/or below the bright 
ones; these are due to imperfections in the grating and we may ignore them. \\

\noindent 
On your trace of the diffraction grating interference pattern, assign
a value for $n$ to each of the dots. Refer to Figure~\ref{fig:diff:diffgratpat}
to see how.  Measure the distances $y_n$ from the $n$th dot to the center 
dot for
all five of the dots. {\bf Note:} Again, the values of $y_n$ for
negative $n$ should be taken to be negative, see 
Figure~\ref{fig:diff:gratgraph}. 


\begin{figure}[htb]
\centerline{\epsfxsize=7cm \epsfbox{diffint_2/gratgraph.eps}}
\caption{An illustration of a typical graph of the grating data.}
\label{fig:diff:gratgraph}
\end{figure}
\vspace*{1cm} 
\noindent 
Record your five $y_n$
vs. $n$ measurements into Table~\ref{tab:DI:Grating}.
\begin{table}[htb]
\begin{center}
\begin{tabular}{|c|c|}
\hline
\multicolumn{2}{|c|}{Measurements of $y_n$ vs. $n$} \\
\hline
Distance ($y_n$) & Order ($n$) \\
\hline
\hspace*{3cm} & \hspace*{3cm}  \\
& \\
\hline
& \\ & \\
\hline
& \\ & \\
\hline
& \\ & \\
\hline
& \\ & \\
\hline
\end{tabular}
\end{center}
\caption{$y_n$ vs. $n$ measurements for diffraction grating pattern.}
\label {tab:DI:Grating}
\end{table}
 
\clearpage

\subsection{In-Lab Computer Work}


\subsubsection{Multiple Slit Diffraction/Interference Patterns}

Using the measured values in Table~\ref{tab:DI:twoslit}, calculate
$y_nd/L$ and its uncertainty for one set of data.  
{\bf Show all work} in the space provided
below. \\
\vspace*{4cm} \\
\noindent Make a plot (with error bars) of $y_nd/L$ vs. $n$ 
Find the slope of the weighted best fit line, and record this
below with uncertainty.

\begin{center}
Slope1=~ \rule{3cm}{.1mm}
\end{center}


\subsubsection{Diffraction Grating Pattern}

Using the measured values in Table~\ref{tab:DI:Grating}, calculate
$y_nd/ \sqrt{y_n^2 + L^2}$ and its uncertainty for one set of data.  
{\bf Show all work} in the space provided below. \\
\vspace*{5cm} \\

\noindent Now, plot $y_nd/ \sqrt{y_n^2 + L^2}$ vs. $n$ with error bars. 
Find the
slope of the weighted best fit line, and record this below with
uncertainty.

\begin{center}
Slope2=~ \rule{3cm}{.1mm}
\end{center}

\subsection{Pre-Classroom Check List}
$\bigcirc$ \hspace*{1cm} One partner has two labeled tracings from single slit \\
$\bigcirc$ \hspace*{1cm} One partner has a labeled tracing from two-slit \\
$\bigcirc$ \hspace*{1cm} One partner has a labeled tracing from grating \\
$\bigcirc$ \hspace*{1cm} Table~\ref{tab:DI:twoslit} completed with units and uncertainties \\
$\bigcirc$ \hspace*{1cm} Table~\ref{tab:DI:Grating} completed with units and uncertainties \\
$\bigcirc$ \hspace*{1cm} $S_1$ with uncertainty and units \\
$\bigcirc$ \hspace*{1cm} $S_2$ with uncertainty and units \\
$\bigcirc$ \hspace*{1cm} 2 Plots labeled completely and correctly \\
$\bigcirc$ \hspace*{1cm} Each student has her/his own plots and worksheet \\
 


\subsection{In-Classroom Calculations \& Analysis}


From your value for Slope1, determine the value of the wavelength of
the laser light for your two-slit interference pattern, $\lambda _1,$
with uncertainty. \\
\vspace*{2cm} \\
\begin{center}
$\lambda _1=$~ \rule{3cm}{.1mm} 
\end{center}
\noindent From your value of Slope2, determine the value of
the wavelength of the laser light for your diffraction grating
interference pattern, $\lambda _2$ with uncertainty. \\
\vspace*{2cm} \\
\begin{center}
$\lambda _2=$~ \rule{3cm}{.1mm}
\end{center}

\noindent
From $\lambda _1$ and $\lambda _2$ calculate the average wavelength
with standard deviation. {\bf Note:} Use equations (0.1) and (0.2) in
$\S$~\ref{sec:intro:uncert}. Record the average value $\lambda
_{avg}$ below with uncertainty. \\
\vspace*{3cm} \\ 
\begin{center}
$\lambda _{avg}=$~ \rule{3cm}{.1mm}
\end{center}

\subsection{In-Classroom Discussion}
\subsubsection{Single Slit Diffraction}
Refer to the traces you made of the single-slit diffraction patterns.
Is the small angle approximation valid for analyzing these traces?
Explain.
\vspace*{2cm}

\noindent 
What happened to the width of the spots in the single-slit diffraction
pattern as you increased the slit width?
\vspace*{2cm}

\noindent
What happened to the separation between the spots in the single-slit
diffraction pattern as you increased the slit-width?
\vspace*{2cm}

\noindent
Do your answers to the above two questions agree qualitatively with
the results obtained in 
$\S$~\ref{sec:diff:singleslit}? Explain by citing the relevant results. \\
\vspace*{2cm} \\

\subsubsection{Multiple Slit Diffraction/Interference Patterns}
What is responsible for the existence of missing orders in a multiple
slit interference pattern? (Hint: examine 
Figure~\ref{fig:diff:multislit diffraction}).  
\vspace*{1.4cm} 

\noindent
Is the small-angle approximation valid for analyzing your trace of the
two-slit interference pattern? Explain by comparing your largest
$y_n$ value to $L.$
\vspace*{3cm}

\noindent
Is your plot of $y_nd/L$ vs. $n$ linear?
\vspace*{.3cm}

\noindent
Compare your value of $\lambda _1$ to the known wavelength of He-Ne
light, 632.8 nm.
\vspace*{2.5cm} 

\noindent
Does your observation of how the sizes of the primary maxima behave as
you increase the number of slits agree or disagree with the discussion
in $\S$~\ref{sec:diff:multislit}? Explain.
\vspace*{3cm}
  

\noindent
Referring to your trace of the two-slit interference pattern and the
qualitative observations you made about the 3, 4, and 5 slit
interference patterns, write an equation that relates the number of
slits, $N$, to the number of secondary maxima, $M,$ that occur in the
corresponding interference pattern. \\
\vspace*{2cm}


\subsubsection{Diffraction Grating Pattern}
Is your plot of $y_nd/ \sqrt{y_n^2 + L^2}$ linear?
\vspace*{.3cm}

\noindent
Compare your value of $\lambda _2$ to the known wavelength of He-Ne
light, 632.8 nm.
\vspace*{1.4 cm}

\newpage


\noindent
Compare your value for $\lambda _{avg}$ with the known wavelength.
\vspace*{2.4cm}

\noindent
Which of the two methods of measuring wavelength is the most {\it
accurate}?
\vspace*{2cm}

\noindent
Is this expected? Why?
\vspace*{1cm}

\noindent
Do you gain anything by having made two separate measurements? What?
\vspace*{1.4cm} 

\newpage
\subsection{In-Classroom Conclusion}

Write a {\it brief} (that is, a one or two paragraph) conclusion for
this lab. In it, you should summarize the physical
principles which were meant to be illustrated in this experiment. You
should also describe the degree to which your data supported these
principles.


\vfill
\noindent Attach plots and tracings to the worksheet. \\
\ \\
{\Large End Worksheet} 

% Go back to ordinary section numbering
\renewcommand{\thesection}{\thechapter.\arabic{section}}

