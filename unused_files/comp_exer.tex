\documentclass[12pt]{article}
\oddsidemargin 0mm
\evensidemargin 0mm
\textwidth=160mm
\textheight=230mm
\headsep=0cm
\parindent=10mm
\headheight=-15mm
\usepackage{epsf}
\begin{document}
\pagestyle{empty}
\begin{center}
{\LARGE \bf PHY 103N} \\
\vspace*{3mm}
{\large \bf Using KaleidaGraph for Data Analysis}
\end{center}

\section{Introduction}
\subsection{What KaleidaGraph Does and Doesn't}

The software package known as KaleidaGraph can be a useful tool for data
analysis. Of course, it will only be useful if you learn how to tell it to do what you
want it to do. This knowledge is best acquired by experience with using the
software, and you will get plenty of that in the upcoming semester. That fact
doesn't help you at the moment, though. Getting started is the difficult part.
Hopefully, this handout will you help you with that. 

This handout is not meant to be a detailed guide to using KaleidaGraph. Instead,
it hopes to demonstrate how to use those features of KaleidaGraph which will be
most useful to you: plotting and fitting curves to data. To accomplish this, each
section of this handout will have two parts. In the first part of each section, the
steps you need to follow to accomplish a basic task are listed. In the second, an
example to illustrate these steps is given. You should follow along with the examples
to get the feel of working with the software. 

When you are using KaleidaGraph to analyze real data, you'll probably have to
perform most of the tasks described below. In that spirit, we'll connect all the
examples with

\subsection{An Hypothetical Experiment.}

Suppose an experiment has been performed that tests the well-known conjecture that the
probablity that a slice of toast buttered on one side will fall butter-side-up is
inversely proportional to the value of the carpet on which it falls. This assertion
can be represented by an equation:

$$ U = a{1 \over V}, $$
where $U$ represents the probabilty of a piece of bread falling butter side up, $a$ is a
proportionality constant, and $V$ is the price of the carpet in U.S. dollars. If you
were to somehow measure the probabilities of the toast falling butter-side-up on several
carpets, and plot these against the reciprocal of the price of the carpet, you'd
expect to see a straight line, with a slope equal to the proportionality constant.
The following examples will lead you through the process of plotting data taken in such an experiment and finding the slope of the best fit line of the plot.

Now for some details. On each run of the experiment, a helicopter dropped 100,000 slices
of buttered toast on a large sample of carpet. Then, the U.S. Dept. of Parks and
Wildlife flew in with their own helicopter to take an aerial photograph of the
result. From the photograph, the fraction of butter-side-up slices was measured, and
reported as the probablility.\footnote{These results have met with some controversy. The Association of Premium Carpet Manufacturers has filed a lawsuit claiming that the results have been altered to make their products seem less desirable.} Only five runs were accomplished before funding ran
out. The results of the experiment are given in the next section.

\section {Entering Data}
\subsection {Actually Entering Data}

KaleidaGraph holds data in a {\bf data window.} One of these should appear when you
start; it's default name is {\bf Data 1.} To enter data,

\noindent
\begin{enumerate}
\item Activate the data window by clicking on it.
\item Position cursor on cell you want to enter data into.
\item Type in a piece of data
\item Move to another cell using mouse, arrow keys, Tab or Return
\end{enumerate}
\indent

Note that KaleidaGraph plots data in a column vs. column fashion, so the
x-coordinate and y-coordinate of a single piece of data should be placed in the same
row, but different columns.
\subsection{Renaming Columns of Data}

This might not seem important at first, but KaleidaGraph labels the axes on plots with
the name of the columns it used in the plot. The default names of columns are ``A",
``B", ``C", etc. These appear in the {\bf column title} row of the data window, along
with a number. To change the name of the column,

\noindent
\begin{enumerate}
\item Double-click on the column title you wish to change.
\item In the ``Column Format:" dialog box which appears there will be a list of column 
titles. Highlight the title you wish to alter by clicking on it.
\item Type in the new title of the column.
\item When you are finished changing names, click on the button labeled 
{\bf Done.}
\end{enumerate}
\indent

The number of any column can be set to zero merely by clicking on the title cell of
that column. The columns to the right then take the numbers 1, 2, 3.... The columns to
the left become unnumbered.

\subsection{Entering Buttered Toast Data} 

The results of the buttered toast experiment are summarized in the following table.
 
\begin{center}
\begin{tabular}{|c|c|}
\hline
$V (\$)$ & $U$ \\
$100 \pm 1$ & $0.43 \pm 0.01$ \\    
$250 \pm 1$ & $0.15 \pm 0.01$ \\ 
$500 \pm 2$ & $0.09 \pm 0.01$ \\
$750 \pm 2$ & $0.06 \pm 0.01$ \\ 
$1000 \pm 3$ & $0.04 \pm 0.01$ \\
\hline
\end{tabular}
\end{center}  

Enter the values for $V$ in the first five rows of column ``A," the uncertainties in $V$
in column ``B," the values for $U$ in column ``C," and the uncertainties in $U$ in
column ``D." Make sure that all the values belonging to a single data point are in the
same row.

Now, retitle the columns by clicking on the title cell of one the columns. Highlight
``A" in the list of column names, and type in ``$V (\$).$'' Don't forget the
units! Now click on ``B" in the list of column titles, and type in ``d$V.$ (\$)''
Continue similarly with the remaining columns.

\section{Operating on Data}
\subsection{Using Formulas}

Often, the raw data you enter is not immediately in the form you need for plotting.
Never fear, KaleidaGraph is capable of performing mathematical operations on the
numbers you have entered. One way of using this feature is to define a formula for
KaleidaGraph. Formulas tell KaleidaGraph to put in one column the result of operations
on data in other columns.

The syntax of formulas you define should be
\begin{center}
c$x$ = f(c$y$, c$z$, ...)
\end{center}
where $x, y,$ and $z$ are the numbers of the columns which contain the numbers you
wish to operate on, and f(...) is the mathematical expression you wish KaleidaGraph to
calculate.

For example, to tell KaleidaGraph to take the fourth power of data in column ``B,''
and place the results four columns to the left, you would first click on the title box
of column ``B," to make it column 0, and then enter the formula c4 = c0 \char'136 4.

\subsection{Entering Formulas}

To actually enter and execute a formula, 

\noindent
\begin{enumerate}
\item Activate the appropriate data window by clicking on it.
\item From the {\bf Windows} menu at the top of the screen, select the option {\it
Formula Entry.}
\item In the ``Formula Entry'' window which appears, click on one of the buttons 
labeled {\bf F1 - F8.}
\item Type in the formula in the space provided in the ``Formula Entry'' window.
\item Click button marked Run in the ``Formula Entry'' window.
\end{enumerate}
\indent

The button labeled {\bf F1-F8} you choose corresponds to one of the function keys at the top
of the keyboard. Pressing that key will bring up the ``Formula Entry''
window again, this time containing the formula you defined.

\subsection{Turning Dollars Into Reciprocal Dollars}
 
Since you want to plot $U$ vs. $1/V,$ you need to make a column containing the
reciprocal dollar values of the carpets. So begin by clicking on the title cell of the
column you've already titled $V (\$).$ This makes it column zero. Move the cursor over to
columns 4 and 5 to make sure they exist, then choose {\it Formula Entry}
from the {\bf Windows} menu. Click on the button marked {\bf F3}, and enter this formula
\begin{center}
c4 = 1/c0.
\end{center}
Now click the {\bf Run} button and watch the magic. You should change the title of this
column now (and don't forget the units!)

Of course, the uncertainties in $1/V$ are different from the uncertainties in $V.$ So
you'll need to get KaleidaGraph to calculate an uncertainty column for you. Try to
figure out the necessary formula yourself.

\section{Plotting Data}

\subsection{Making a Scatter Plot}
These two sections on plotting and fitting data require little motivation. But a few
important notes will be made. First, {\bf Never} connect-the-dots when you plot data.
Fortunately, if you follow these steps, you'll never forget and accidentally do it.

\noindent
\begin{enumerate}
\item Activate the window containing the data you want to plot by clicking on it.
\item From the {\bf Gallery} menu at the top of the screen, select the {\bf Linear}
submenu, and from that select the {\bf Scatter} option.
\item A dialog box will appear. In it, there will be columns of circles labelled ``X"
and ``Y." Under ``X" click on the circle in the row containing the title of the column
which contains the x-coordinates of your data. A solid black circle should appear.
\item Do the same with your y-coordinates in the column labelled ``Y."
\item Click on the button labelled {\bf New Plot.}
\end{enumerate}
\indent

Note the conspicuous lack of error bars.

\subsection{Adding Error Bars}

\noindent
\begin{enumerate}
\item Activate the window containing your plot by clicking on it.
\item From the {\bf Plot} menu at the top of the screen, select the option ``Error
bars..."
\item In the ``Error Bar Variables" dialog box which appears, click on the square
labeled ``X Err.''
\item Click and hold on one of the two rectangles labeled ``\% of values," and select
the option ``Data Column." 
\item Select the column which contains the uncertainties in the x-coordinates of your
plotted data. 
\item Click on button labelled {\bf OK.}
\item Now, follow the same procedure starting with the square labelled ``Y Err.''
\item Click on button labelled {\bf Plot.}
\end{enumerate}
\indent

\subsection{Plotting Buttered Toast vs. Inverse Dollar Value}

Simply follow the above instructions to make a scatter plot, choosing $1/V$ as your
x-coordinate column, $U$ as your y-coordinate column. For error bars, choose the
appropriate columns. You're almost there.

\section{Performing a Weighted Least-Squares Fit on Plotted Data}
\subsection{Ditto}

We come to the moment of truth. Both the slope, and y-intercept of a linear plot are
often important pieces of information to obtain. Of course, they are meaningless without
uncertainties. Therefore, you should make sure you take into account the uncertainties in your
points when calculating the fit. 

\noindent
\begin{enumerate}
\item Activate the window containing your plot by clicking on it.
\item From the {\bf Curve Fit} Menu at the top of the screen, select the {\bf General}
submenu, and from that, select the option {\bf fit1.}
\item In the ``Curve Fit Selections:" dialog box which appears, click on the button
labelled {\bf Define...}
\item In the new dialog box that appears, click on the square labelled ``Weight Data"
so that an X appears in it.
\item Click on the button labelled {\bf OK.}
\item In the ``Curve Fit Selections:" dialog box, click on the square next to the
column title which contains the y-coordinate of the data you are plotting.
\item A new dialog box will appear called ``Weight Data From Column:". By clicking on
the buttons labelled $\ll$ and $\gg$ , make sure the name of the column containing the
uncertainties for the y-coordinate appears in the window.
\item Click on the button labelled {\bf OK}.
\item Now click on this button labelled {\bf OK}.
\end{enumerate}
\indent

Now you need the numerical results of the fit. Simply choose the ``Display Equations"
option from the {\bf Plot} menu, and a table containing the numbers you desire will
appear. Note that, in this table, m1 is the y-intercept and m2 is the slope of the
best fit line.

\subsection{Finding the Buttered Toast Constant}

Again, simply follow the above steps, using $U$ as your y-coordinate and the delta$U$ 
column as the column of the weights. Print out your plot, making sure you report the
slope as the constant of propotionality in the correct units.
  
\subsection{The Work You Should Turn In}

After you have followed through the above example, you should turn in the printout you have made. This printout should at least contain: properly labeled axes, x and y error bars, and the best fit line plotted by KaleidaGraph through your plotted points. Also, somewhere on the page below the plot, you should report {\it in a complete sentence} what you have found to be the constant of proportionality (with uncertainty!).

Of course, your TA may require more of you.
\end{document}




















