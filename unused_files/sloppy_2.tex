\section{Procedure}

For most of the procedures in this lab, we will require a distance of at least
2~m between the slit assembly and the screen you'll use to observe the pattern,
$L$ must be large for the small angle approximation.  As a screen, we'll 
use large sheets of computer paper taped to the wall; these are nice, because 
you can easily trace the pattern on them.  However, this means that several 
groups of students will have to aim the lasers {\it across} the room.  As long
as everyone is careful not to aim their own lasers up at eye level and we 
remain aware of everyone else's laser's, there should be no problems.  Be 
patient and understanding if your instructor or a classmate needs to walk in
front of your beam when you're in the middle of making a sketch.

\subsection{Single Slit Diffraction}  

Find a convenient way to aim the laser at one of the walls so that there's
no electrical outlet or other obstruction to taping your paper screen onto the
wall; make sure that you will have a 2~m seperation between the {\it slit 
assembly} and the wall.  Place the single slit apparatus in front of the laser 
and align the slit with the beam. This may take a bit of playing around, try 
placing the slit apparatus on its side or prop it up if necessary, you might 
also need to adjust the slit width; just be careful not to look directly into 
the laser beam or reflections.  When you have the slit set up properly you 
should see a diffraction pattern as shown in {\bf Figure}. Using a length
of string and a meterstick, measure the distance from the slit to the screen.
Examine the effect of changing the slit width on the pattern.  What happens to 
the width of the spots as you increase the slit width? What about the 
separation between the spots? Trace two patterns: one with a relatively large 
slit width and another with a relatively small slit width; make sure that you 
label these appropriately.   

\subsection{Multiple Slit Diffraction/Interference Patterns} 
\label{sec:diff:multislit}

Replace the single slit assembly with the mulit-slit slide (mounted in a 
magnetic holder) and remeasure the slit to screen distance.  Adjust the slide
so that the two slit pattern (slit pattern, not interference pattern) is 
directly in front of the beam; adjust it until you get a clear interference 
pattern on the screen. Make sure that you write down the slit width and slit 
spacing values that are printed on the slide. Trace the pattern (remember to 
label it).  What is the number of secondary maxima?  Can you identify missing 
orders? Measure the distances $y_n$ for at least 8 primary maxima; refer to 
{\bf Figure} as an aid.
%
%\epsfxsize=13cm \epsfbox{diffint/interpatt.eps}\\
%
Compare $y_n$ with $L$, is the small angle approximation valid? Plot $y_n d/L$ 
vs.~$n$, is this linear? Find the slope of the best line fit with uncertainty;
use it to calculate the wavelength of the laser light (with uncertainty).
How does this compare with the known wavelength of 632.8~nm for He-Ne laser
light?

Let's also examine the qualitative properties of the other slit patterns. 
Observe the patterns for the 3, 4, and 5 slit patterns, you don't need to 
sketch them.  How many secondary maxima appear for each pattern? Is this what
we expect?  What happens to the size of the primary maxima as the number of 
slits increases?

\subsection{Diffraction Grating Pattern} 

Replace the mult-slit slide with the diffraction grating slide.  Can you see
the diffraction pattern?  What if you place a sheet of paper directly in front
of the grating?  Move the laser and grating so that you decrease the distance
between the grating and the wall. Adjust the grating to screen distance until 
you can see five dots, as illustrated in {\bf Figure}. 
%
%\hspace*{3cm} \epsfxsize=7cm \epsfbox{diffint/diffpatt.eps}\\
%
Measure the grating to screen distance.  Trace the 5 bright spots. Some dim 
spots may appear above and/or below the bright ones; these are due to 
imperfections in the grating and we may ignore them. Measure the $y_n$ for all
of the spots and make the assignments of $n$ to them shown in {\bf Figure}.  
Compare the values of $y_n$ with $L$, is the small angle approximation valid?
Calculate $y_n d/\sqrt{y_n^2 + L^2}$ for each data point and plot these 
vs.~$n$; is this graph linear? This should look like {\bf Figure} 
% 
%\hspace*{3cm} \epsfxsize=7cm \epsfbox{diffint/gratgraph.eps}\\
%
Calculate the slope of a best line fit to this data (with uncertainty) and use
it to calculate $\lambda$, with uncertainty.  Compare this to the known 
wavelength and also to the wavelength measured in 
part~\ref{sec:diff:multislit}. Using both of your measured values, calculate 
the average wavelength and standard deviation; compare this to the known 
value.  Which method has the greater {\it accuracy}? Is there any reason
you might expect this? Do you gain anything by having made two seperate 
measurements? Discuss in detail.  

