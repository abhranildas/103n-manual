Welcome to Physics 103N, Engineering Physics II Lab. This class is the 
continuation of Physics 103M, and is a corequisite to Physics 303L, 
Engineering Physics II Lecture. Although this class is a corequisite to 
Physics 303L, the topics we discuss here are not necessarily exactly those 
discussed in lecture. There are several reasons for this: the first is that 
timing the labs with the lectures is impossible; second, you don't always 
need a detailed theoretical description of phenomena to measure and 
characterize their properties. It is this empirical approach that we want to
emphasize here. Third, there are important physical phenomena that are not 
covered in detail in the lecture, because of a lack of time. We will examine 
some of these in this course. So, don't expect a mere repeat of the lectures 
here.

There are two essential reasons for this course. First, it should give you 
some general background knowledge of how experimental work is actually done. 
You will learn how to use equipment such as multimeters, frequency generators,
and oscilloscopes among others. Further, you will see how to measure various 
properties of electronic circuits and optical systems. These are all very 
practical skills. Secondly, it should help you see that all the conjectures 
and calculations that you learn about in lecture do describe events in the 
real world. You will quantitatively verify some of the formulas derived in
the lecture to check the professor and make sure you haven't been lied to. If 
not, then you will probably believe what else is said in lecture, whereas if 
you've been told lies, that makes everything else the professor expounds is
liable to suspicion. So be on the lookout for discrepancies!

Most of the equipment you need will be provided in lab.  You should bring a 
pen and pencil (sketches should always be done in pencil) and paper, a scientific calculator ({\it i.e.}, one with logs and trigonometric 
functions, not necessarily a graphing calculator), and this 
manual to each lab 
meeting.  You might want to keep a notebook instead of writing observations
and calculations on loose paper.  A notebook will help you to keep organized 
so that you don't lose your notes or confuse your data.  Your reports will be 
turned in on the worksheets printed in this manual.  So if you do use a 
notebook, make sure it has perforated sheets, as you will turn in any extra 
sheets with your worksheet.  In any case, 
{\it avoid} the hardbound laboratory notebooks 
(the ones with the carbon paper), since they are unnecessarily expensive 
($>$\$10).  We expect that you have the textbook assigned to the 303L lecture 
course available; the reference is
\begin{quote}
R.\ A.\ Serway and J.\ W.\ Jewitt, {\it Physics for Scientists \& Engineers}, 6th edition 
(updated), Thomson Publishing, Belmont, CA (2004). 
\end{quote}
An additional reference, that we'll refer to repeatedly in our discussion of
error analysis in Chapter~\ref{ch:intro}, but is by no means required reading,
is
\begin{quote}
P.\ R.\ Bevington and D.K.\ Robinson, {\it Data Reduction and Error Analysis 
for the Physical Sciences}, 2nd.\ edition, McGraw-Hill, Inc., New York (1992).
\end{quote}

On a final note, we add that this is a new lab manual, and as such, is just now
meeting the tests and demands of students. Some typos, ambiguities, or other
inadequacies are bound to have slipped our grasp.  Please bring any errors or
confusing parts of the manual to the attention of your instructor.  Student
input is invaluable to the production of a document that students depend on for
learning.  As an alternative, feel free to E-mail your comments and suggestions
to 103n@physics.utexas.edu. \\ 
% No headings for this page.
\thispagestyle{empty}
%
\vfill

