A particle of electric charge $q$ and mass $m$ in a constant electric 
potential $V$, will be accelerated linearly, since the force on the charge is
\beq
\vec{F}_{\mbox{\tiny elec}} = -q \frac{dV}{dx} \hat{x},
\eeq
where $\hat{x}$ is the unit vector in the direction of the acceleration on the
electron. The work done on the charge by the potential is 
\beq
W_{\mbox{\tiny elec}} = \int \vec{F}_{\mbox{\tiny elec}} \cdot d\vec{x}= qV.
\eeq
This work done increases the kinetic energy of the charge, so that if we ignore
any initial velocity it had when it left the filament, we find that the 
kinetic energy of the charge is 
\beq
\frac{1}{2} mv^2=qV. \label{eq:ed:kinetic}
\eeq 
We can solve this to determine the velocity $v$ from the charge, mass, and 
potential
\beq
v=\sqrt{\frac{2qV}{m}}. \label{eq:ed:velocity}
\eeq

The charge now enters the uniform magnetic field at $x=z=0$ with velocity
\beq
\vec{v}_{\mbox{\tiny init}}=\sqrt{\frac{2qV}{m}}~\hat{x} \label{eq:ed:initvel}
\eeq 
(note the coordinates drawn in Figure~\ref{fig:ed:elec}). We take the 
magnetic field to be in the $y$-direction, $\vec{B}=B\hat{y}$, where the 
magnitude of the field $B$ is constant. This means that the magnetic field is 
pointing out of the page in Figure~\ref{fig:ed:elec}. The force on the charge 
due to the magnetic field is given by the Lorentz force law  
\beq
\vec{F}_{\mbox{\tiny mag}} = q\vec{v}\times\vec{B}. 
\eeq
If we use dots to denote time derivatives, as in $\vec{v} = \dot{\vec{x}}$, we 
have from Newton's second law
\beq
m \ddot{\vec{x}} = q \dot{\vec{x}} \times \vec{B},
\eeq
or, carrying out the cross product and using $\vec{x}=(x,y,z)$
\beq
m\left(\ddot{x}\hat{x} + \ddot{y}\hat{y} + \ddot{z}\hat{z} \right) 
= qB \left(\dot{x}\hat{z} - \dot{z}\hat{x}\right).
\eeq
If we equate vector components, we have the set of equations
\beq
m\ddot{x}=-qB\dot{z}, \label{eq:ed:xcomp} 
\eeq
\beq
\ddot{y} = 0, \label{eq:ed:ycomp} 
\eeq
\beq
m\ddot{z}=qB\dot{x}. \label{eq:ed:zcomp} 
\eeq 
Equation~(\ref{eq:ed:zcomp}) tells us that the force in the $y$ direction is
zero. Equations (\ref{eq:ed:xcomp}) and (\ref{eq:ed:zcomp}) can be integrated
once with respect to time to give 
\beqra
m\dot{x} &=& -qBz + c_1, \nonumber \\
m\dot{z} &=& qBx + c_2, \label{eq:ed:intonce} 
\eeqra
where $c_1,c_2$ are constants of integration. We can solve for the constants by
using the initial condition (\ref{eq:ed:initvel}). When $z=0$, 
$\dot{x}=\sqrt{2qV/m}$, so that $c_1=\sqrt{2qV/m}$. When $x=0$, $\dot{z}=0$, so
$c_2=0$. Substituing (\ref{eq:ed:intonce}) into (\ref{eq:ed:xcomp}) and 
(\ref{eq:ed:zcomp}) yield two equations
\beqra
\ddot{x} &=& -\omega^2 x \nonumber \\
\ddot{z} &=& -\omega^2 z -\omega^2  \sqrt{\frac{2mV}{qB^2}},
\eeqra
where we have defined $\omega=qB/m$. These are just equations of harmonic 
motion, so they have the solution
\beqra
x &=& A \sin(\omega t +\theta) \nonumber \\ 
z &=& -\sqrt{\frac{2mV}{qB^2}} - A \cos(\omega t +\theta),
\eeqra
where $A$ is an amplitude and $\theta$ is a phase angle. We can again
use the initial conditions $x=z=0$ at $t=0$.  When $t=0$, we have from above
$x=A\sin\theta=0$, which is satisfied if $\theta=0$. Then, when $t=0$, 
$z= -\sqrt{2mV/qB^2} - A=0$, so that $A=-\sqrt{2mV/qB^2}$. We therefore have a 
solution
\beqra
x &=& \sqrt{\frac{2mV}{qB^2}} \sin\omega t \nonumber \\ 
z &=& -\sqrt{\frac{2mV}{qB^2}} \left( 1+ \cos\omega t \right), 
\eeqra

If we form the combination
\beqra
x^2 + \left( z+ \sqrt{\frac{2mV}{qB^2}}\right)^2
&=&  \frac{2mV}{qB^2} \left( \sin^2\omega t + \cos^2\omega t \right)
\nonumber \\
&=& \frac{2mV}{qB^2}, \label{eq:ed:circ}
\eeqra
we notice a very interesting thing. The form $(x-a)^2 + (z-b)^2 = R^2$ is that 
of a circle of radius $R$ which is centered at the point $(a,b)$. Applying this
knowledge to (\ref{eq:ed:circ}) tells us that