\documentclass[12pt,letterpaper]{book}

\usepackage{epsf}
\usepackage{graphicx}
\usepackage{epstopdf}
\usepackage[utf8]{inputenc}

\input epsf 

\renewcommand\thefootnote{\fnsymbol{footnote}}
\setcounter{footnote}{1}

\title{{\huge Electromagnetism and Optics} \\ 
\ \\
{\Large \bf The Lab Manual for PHY 103N \\
Engineering Physics II Laboratory }}
\author{ Department of Physics \\ 
 University of Texas at Austin, Austin, TX 78712\\
2005-2006}

\date{\today}

\begin{document}

\maketitle

% number frontmatter (including preface) with small roman numerals
\renewcommand{\thepage}{\roman{page}}

\tableofcontents

% call the preface ``Preface''
\renewcommand{\chaptername}{}
\renewcommand{\thechapter}{}

\chapter{Preface}  % this alone will give the preface
\input 0_intro/preface.tex

% Go back to normal chapter style
\renewcommand{\chaptername}{Chapter}

% Go back to arabic page numbering
\renewcommand{\thepage}{\arabic{page}}
\setcounter{page}{0}

% The introduction will be chapter 0. 
\renewcommand{\thechapter}{0}

\chapter{Introduction}
\label{ch:intro}
\input 0_intro/0_intro.tex

% Back to ordinary arabic numbering for the labs
\renewcommand{\thechapter}{\arabic{chapter}}
\setcounter{chapter}{0}


\chapter{Electrostatics}
\input 1_electrostatics/1_electrostatics.tex

\chapter{DC Circuits}
\input 2_dc/dc.tex

\chapter{Electron Dynamics}
\input 3_electrondynamics/3_electrondynamics.tex

\chapter{Measurements with the Oscilloscope}
\input 4_oscilloscope/4_oscilloscope.tex


\chapter{RC Circuits and Filters}
\input 5_rccircuits/5_rccircuits.tex

% \chapter{Electromagnetic Induction}
% \input induction_2/induction_2.tex

% \chapter{Polarization of Light}
% \label{ch:pol}
% \input polarization_2/polarization_2.tex

% \chapter{Refraction Optics}
% \label{ch:optics}
% \input optics_2/optics_2.tex

% \chapter{Imaging Optics}
% \label{ch:imaging}
% \input optics_2/imaging_2.tex

% \chapter{Diffraction and Interference of Light}
% \input diffint_2/diffint_new.tex




    
\end{document}