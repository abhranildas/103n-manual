\documentclass[12pt]{article}
\oddsidemargin 0mm
\evensidemargin 0mm
\textwidth=160mm
\textheight=230mm
\headsep=0cm
\parindent=10mm
\headheight=-15mm
\usepackage{epsf}

\begin{document}

\begin{center}
{\LARGE \bf PHY 103N - RC Circuits} \\
\vspace*{3mm}
{\large \bf Procedural Modifications}
\end{center}

\noindent
{\bf Instructions}: 
This handout outlines several changes to the procedures in
the lab manual. Where this handout and the manual conflict, follow the 
instructions contained here.

\section{Apparatus}

We will use the T-connectors to split the output of the function generator, as
illustrated in Figure~\ref{fig:wiring}.
\begin{figure}[htb]
\epsfxsize=8cm
\centerline{\epsfbox{wiring.eps}}
\caption{We will use the T-connector and trigger the oscilloscope on 
channel~1.}
\label{fig:wiring}
\end{figure}
The split signal will allow us to trigger on the raw output of the function 
generator (on channel~1).

We will use a 51~$\Omega$ resistor and a 68~nF capacitor in our RC~circuit.
Since the function generator has a 600~$\Omega$ output impedance, what should
really go into our formulae during our analysis is the equivalent resistance 
of each circuit. Take care when evaluating this, since the resistor and the
function generator can be in series or in parallel, depending on where you 
decide to make measurements. The circuit we will set up on the board is 
illustrated in Figure~\ref{fig:rcboard}.
\begin{figure}[htb]
\epsfxsize=15cm
\centerline{\epsfbox{rcboard.eps}}
\caption{The proper breadboard connections for our RC circuit. Note that the
oscilloscope can be connected so that it measures the voltage across the 
capacitor ($V_C$) or that across the resistor ($V_R$).}
\label{fig:rcboard}
\end{figure}
\clearpage

\section{DC Response}
Arrange the circuit in Figure~\ref{fig:rcboard} on the breadboard and set the 
function generator to provide a square wave at about 1~kHz.  Connect the 
alligator clips of the BNC-to-alligator clip wire to jumpers on the board 
so that they are measuring the voltage across the {\it capacitor}, $V_C$; 
send this into channel~2 of the oscilloscope. Trigger the oscilloscope on 
channel~1 and display channel~2; adjust the scope until
you can see the decay/growth response illustrated as ``output'' at the bottom
of Figure~4.9 in the manual.  By adjusting the V/div and sec/div knobs of the
scope, focus in on a {\it decay} portion of the signal. Adjust the channel~2
vertical position and the horizontal position knobs until the maximum of the 
decay curve lands on a vertical grid line (voltage axis) and the minimum lands 
on a horizontal grid line (time axis), as illustrated in 
Figure~\ref{fig:scope}.
\begin{figure}[htb]
\epsfxsize=8cm
\centerline{\epsfbox{scope.eps}}
\caption{Adjust the scope until you have the decay curve aligned with a 
``voltage axis'' and ``time axis.''}
\label{fig:scope}
\end{figure}
Now use the axes you thus created to measure the voltage-time pairs referred to
in the manual. Perform all of the analysis on this data as asked in 
Section~4.4.1 of the manual.
%\clearpage

\section{AC Response}
Now adjust the function generator to provide a sinusoidal input to the circuit.
Referring to the circuit diagram in Figure~\ref{fig:lowpass} for this 
arrangement, calculate the equivalent resistance from the nominal value of 
resistance and the output impedance of the function generator. 
\begin{figure}[htb]
\epsfxsize=8cm
\centerline{\epsfbox{lowpass.eps}}
\caption{A low-pass filter arrangement.}
\label{fig:lowpass}
\end{figure}
Use this and the nominal capacitance to calculate the cutoff frequency for the 
circuit. 

Ignore the phase difference measurement. 

Set the function generator to a 
frequency below the cutoff. You can tell that you're below the cutoff when you 
see that the amplitude of the output signal doesn't depend on the frequency;
this will probably be somewhere in the 1~kHz range of the function generator. 
Continue with the amplitude versus frequency measurements and subsequent
analysis in Section 4.4.2 for this circuit. 

Ignore the procedure for measuring voltage across the resistor.

\end{document}


