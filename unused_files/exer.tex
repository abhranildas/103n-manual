\documentstyle[12pt]{article}
\pagestyle{empty}
\begin{document}

\thispagestyle{empty}
\begin{center}
{\large Two Exercises for Kaleidagraph}
\end{center}

\noindent
{\bf 1.} Imagine that you are riding in a car with your uncle Bob, his sister 
Sandy, and your cousins Rob and Gary. As you head down a back road in 
central Florida, a brilliant blue light suddenly bathes the car, and you
fall unconscious. When you wake up, you find yourself in a room with no 
doors and plain grey walls. An unknown source of illumination allows you
to see that your only company is a spark-tape apparatus, familiar to you
from your first semester lab class. You immediately realize that you can
conduct a simple experiment to determine the gravitational acceleration of 
the planet you are on. You conduct the experiment and find the following
distance fallen versus time data:

\vspace*{0.5cm}
\begin{center}
\begin{tabular}{|c|c|}
\hline
\multicolumn{2}{|c|}{Distance $(d)$ vs. time $(t)$ Measurements} \\
\hline\hline
$d$ (m) & $t$ (s) \\
\hline
  $.05 \pm .005$      &   $.1 \pm .01$   \\
\hline    
  $.20 \pm .005$      &   $.2 \pm .01$   \\
\hline
  $.43 \pm .005$     &   $.3 \pm .01$   \\
\hline
  $.79 \pm .005$     &  $.4 \pm .01$    \\
\hline
  $1.21 \pm .01$     &  $.5 \pm .01$    \\
\hline
  $1.74 \pm .01$      & $.6 \pm .01$     \\
\hline
\end{tabular}
\end{center}

\vspace*{0.5cm}
\noindent
Do the following

{\it a)} Plot $d$ vs. $t^2$ and obtain a {\bf weighted} least squares fit.
(Remember, use {\bf fit1} under the General option of the curve fit menu.)
Error bars, properly labeed axes and units are essential!

\vspace*{0.5cm}
{\it b)} Determine the acceleration due to gravity from the results of 
your curve fit.  Report this at the bottom of this page, including
units and uncertainties.

\vspace*{0.5cm}
{\it c)} Try to guess where you are....
\newpage

\noindent
{\bf 2.} In an experiment you will conduct in a few weeks, you will measure
the voltage $(V)$ of a discharging capacitor as a function of time $(t).$
You will be required
to plot 

$$
\ln \left( {V \over V_0} \right) \quad vs. \quad t
$$

\noindent
where $V_0$ is the initial voltage (that is, the voltage measured at time 
$t=0.)$

\noindent
Use the following sample data:


\vspace*{0.5cm}
\begin{center}
\begin{tabular}{|c|c|}
\hline
\multicolumn{2}{|c|}{Voltage $(V)$ vs. time $(t)$ Measurements} \\
\hline\hline
$V$ (V) & $t$ (ms) \\
\hline
  $2.0 \pm .05$      &   $ 0.0 \pm .01$   \\
\hline    
  $1.8 \pm .05$      &   $ 0.5 \pm .01$   \\
\hline
  $1.7 \pm .05$     &   $ 1.1 \pm .01$   \\
\hline
  $1.4 \pm .05$     &  $ 1.6 \pm .01$    \\
\hline
  $1.3 \pm .05$     &  $1.9 \pm .01$    \\
\hline
  $1.2 \pm .05$      & $2.5 \pm .01$     \\
\hline

\end{tabular}
\end{center}

\vspace*{0.5cm}
\noindent
Do the following

{\it a)} Plot $\ln (V / V_0)$ vs. $t$ and obtain a {\bf weighted} least
squares fit. Again, be sure to include error bars, units, and properly
labelled axes.

\vspace*{0.5cm}
{\it b)} Report the slope and intercept of your plot at the bottom of this
page. Units and uncertainties are a must!




\end{document}
