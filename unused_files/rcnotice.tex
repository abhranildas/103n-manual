\documentclass[12pt]{article}
\oddsidemargin 0mm
\evensidemargin 0mm
\textwidth=160mm
\textheight=230mm
\headsep=0cm
\parindent=10mm
\headheight=-15mm
\begin{document}
\pagestyle{empty}

\begin{flushleft}
\begin{tabular}{ll}
To: & 103N TA's (Scott Hawley, Christopher Jones, Kevin Koch, Brian LaCour, \\ 
& Xiao-Rong Morrow, James Pitts, Ed Qubain, Deidre Shoemaker, Rusty Towell, \\
& Tom Yudichak) \\
From: & Richard Corrado \\
Re: & RC Circuits modifications.
\end{tabular}
\end{flushleft}

I was able to get the RC circuit setup to work properly as far as being able
to see both the low-pass and hi-pass filter configurations in action. We still
can't see the pure function generator output while the circuit is connected,
so we'll only use the T-connector to let us trigger off of channel~1 (we don't
really need to do this, but it doesn't really matter\ldots)

I've provided diagrams of both the breadboard configuration and the hi- and
low-pass filter configurations.  You should note that the capacitor is in the
circuit {\it before} the resistor. From some combination of voodoo and spite,
this was the only way I could make both $V_C$ and $V_R$ measurements and still
see the right frequency attenuation.  If the resistor comes before the 
capacitor, you get a flat line for $V_R$~vs.~$\omega$.  This is weird behavior
for an AC circuit and I'd be happy if someone could come up with a decent
explanation of it. For now I'm happy that something works.

I cleared up the discussion of how to make $V$~vs.~$t$ measurements off of the 
decay curve with Figure~3 in the handout. I also gave a vague, but hopefully
useful elaboration of what the Additional Question is asking for.  Feel free
to drop your students a few more hints if you want to.

Let me know how things work out after your first lab; if there's any major 
problems, I'd like to clear them up for the later lab sections, if possible.

\end{document}


