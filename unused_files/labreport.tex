Your lab reports will be graded on both content and presentation. A good 
report is not only factually accurate, but is well organized. This document 
outlines the content and the logical construction that your report should 
contain, please read it carefully.

As a general rule be concise when writing your report, but be clear. Each 
section (with the exception of the title page) should include brief 
introductory remarks,  typically a single sentence, which explain exactly what 
is being performed and to what purpose. {\bf If we use detailed reports, we 
should require abstracts and introductions for those.} This will be elaborated 
on for each section individually, but keep in mind that the report should be 
kept brief. \\
\begin{center}
\begin{tabular}{ll}
{\large \bf Structure:} & Title Page\\
& Data\\
& Analysis\\
& Additional Questions\\
& Discussion
\end{tabular}
\end{center}

\section{Title Page}

This should simply contain your name, the number and title of the experiment,
the date the experiment was performed and the names of any partners
you worked with.  

\section{Data Section} 

This preliminary section includes all data taken and any constant
parameters or sketches needed for later analysis. The original data sheets,  
with the initials of the TA, should be included, even if the data are recopied.
You should make a few remarks (keep them brief) so that the reader knows what 
it is that he's looking at. If you measured voltage vs. current for a resistor,
say this. Don't just write $V$ vs.\ $I$. Long lists of data may be 
appropriately displayed in tabular form.  It is most important that the units 
and an estimate of the experimental uncertainty of each piece of data be 
clearly stated. Since this is the Data Section, no calculations should be 
reported here, with the exception of those needed in order to carry out the 
experiments, such as for the calibration or positioning of a piece of 
equipment. Most calculations are better left to the Analysis Section. 
{\bf We should decide if we want them to carry out all of the analysis for 
each section seperately, following the presentation of recopied data.  The 
original data sheets will still be turned in and the conclusion section should 
be written for the whole lab, i.e.\ not piecemeal.}

As an example, suppose that an experimental procedure consisted of measuring 
the voltage across a 100k$\Omega$ resistor as the current through the
resistor was varied. The uncertainties in each of the voltage values was the 
same as were those of the current. The relevant part of the data section might 
look like this:
\begin{quote}
{\underline{Voltage Measurement}}\\
A variable resistor was used to vary the current through a 100k$\Omega$ 
resistor. The voltage across the resistor was measured as a function of 
current.\\
\begin{center}
\begin{tabular}{l|l}
\multicolumn{1}{c} I(mA)  & \multicolumn{1}{c} V(V)\\
\hline
1.2 $\pm$ 0.1 & 2.43 $\pm$ 0.01 \\
2.5 & 3.67\\
$\vdots$ & $\vdots$\\
\end{tabular}
\end{center}
\end{quote}

There are several items illustrated by this example. Only a brief description 
of the procedure was given, but it is clear what measurements were made. Since
the same measurement was repeated, the data have been presented in a table.
The table is labeled with the quantities being reported (in this case 
by mathematical abbreviations) and their units. Note that even though a $V$ has
been used to label the voltage values, the units of volts (again labeled by a 
$V$) have still been included, despite the fact that the symbol $V$ was 
repeated.  An estimate of the experimental uncertainty has been included for 
the data, but since it is the same for each piece of data, it was unnecessary 
to recopy it on each row of the table. However, note that this will not always 
be the case.        

\section{Analysis Section}

This includes all graphs, calculations, etc.\ used to obtain the physical 
quantities of interest from the data taken in the lab. Your introductory 
remarks should just outline what quantities are being used in the calculation
and what is being calculated.  If you're plotting voltage vs. current values 
to obtain the resistance of a resistor, then say so. Don't just have a $V$ vs.\
$I$ graph and then a calculation which results in some value of $R$. 

Also, since this is the Analysis Section, it is not in good style to begin 
formulating conclusions about the results.  Save all remarks on conclusions 
about your results for the Discussion Section. It is also crucial that care be 
taken in following the units and error propagation through each step of the 
analysis.  Do not ignore units for several steps of a calculation and then 
just write down what seem to be the proper units.  By showing how the units 
follow through a calculation many mistakes can be avoided. 

If you must perform the same calculation several times, you do not need to 
show each calculation.  You should provide a single sample of the calculation 
including error propagation.  Make a table up of the results for the set of 
data. This also applies to calculations you've performed in previous writeups. 
If you need to perform a calculation that was also made in a previous lab you 
don't need to provide me with a sample calculation. {\bf We can give a 
specific example here.} 


Suppose a required  calculation was that of the power absorbed by the 
resistor in the experiment in our previous example. The analysis might look 
like:
\begin{quote}
{\underline{Power Absorbed}}
Measurements of the voltage across and the current through a 100k$\Omega$ 
resistor are used to calculate the power absorbed by the resistor. \\
\ \\
Sample Calculation: 
\begin{eqnarray}
P &=& IV \nonumber \\
&=& (1.2 \mbox{mA})(2.43 \mbox{V}) \nonumber \\
&=& 2.916 \mbox{mW} \nonumber  
\end{eqnarray}
Error Propagation:
\begin{eqnarray}
\frac{\Delta P}{P} &=& \frac{\Delta I}{I} + \frac{\Delta V}{V} \nonumber \\
&=& \frac{0.1 \mbox{mA}}{1.2 \mbox{mA}} + \frac{0.01 \mbox{V}}{2.43 \mbox{V}}
\nonumber \\  
&=& 0.08744856 \nonumber \\
\Delta P &=& 0.08744856 * 2.916 \mbox{mW} \nonumber \\ 
&=& 0.255 \mbox{mW} \nonumber \\
P &=& 2.92 \pm 0.26 \mbox{mW} \nonumber
\end{eqnarray}
Results: \\ 
\begin{center}
\begin{tabular}{l|l|l}
\multicolumn{1}{c} I(mA)  & \multicolumn{1}{c} V(V) &
\multicolumn{1}{c} P(mW)  \\
\hline
1.2 $\pm$ 0.1 & 2.43 $\pm$ 0.01 & 2.92 $\pm$ 0.26\\
2.5 & 3.67 & 9.18 $\pm$ 0.39\\
$\vdots$ & $\vdots$ & $\vdots$\\
\end{tabular}
\end{center}
\end{quote}

Note that a summary sentence introduced the section and that the relevant 
equations to be used for the calculation were clearly presented.  Enough
of the calculation was presented so that each part could be followed: values
were substituted into the equations, but since a calculator was used, most
of the intermediate steps could be left out for brevity. The units of all 
quantities were always included and the concise mW was used instead of 
$10^{-3}$ W. The uncertainty was calculated and used to provide the correct 
number of significant figures, but as the table shows, a separate uncertainty 
calculation was necessary for each value, since the uncertainty varied with
the data.   

\section{Additional Questions}
	
There will often be additional questions about the physical  principles
or experimental techniques used in an experiment.  Answers will usually consist
of a brief calculation or discussion,  or possibly a sketch.  The objective of
such questions will be mainly to apply the understanding of the phenomenon
examined to either a ``real world'' case or to a slight variation in the
experiment that was not performed in the lab.  Occasionally, an additional 
question might address a possible source of error for one of the procedures in
a lab.  In any case, it is a good idea to work through the additional questions
before writing up your discussion section. 

\section{Discussion Section}
	
Here you will make specific statements about the results of the
analysis and draw conclusions regarding the accuracy of techniques and success 
in reaching the objectives of the lab.  You should only briefly allude to the 
theory which applies to the experiment performed,  it is often sufficient to 
state what principles apply and to quote the specific equations or accepted
result which is being examined.  You should state clearly whether or not your
graphs have the expected functional form ({\it i.e.}\ linear, etc.)\ and 
whether or not your calculated results agree within the calculated 
uncertainties to any accepted values given in the lab manual.  You should 
attempt to explain any discrepancies in terms of specific areas of the 
experimental technique which could have been subject to error.  Even if your 
results do agree with expectations, you should briefly point out such sources 
of possible error,  as this is one way of demonstrating your understanding of 
the experiment you have performed.  Note that you should never claim ``human 
error'' as a source of error.  If you believe there was a systematic error 
which occurred during the data taking, you may say so, but you must explain 
exactly which part of the apparatus or procedure was involved. Note also, that 
an ``error in the calculations'' is not a source of error either.  If you 
believe you've made a mistake when performing your analysis and you can't find 
it, go to your TA's office hours and let him or her find it for you.   

