\vfill
\pagebreak

%  Label worksheets by \thechapter.W
\renewcommand{\thesection}{\thechapter.W}


\section{RC Circuits and Filters Worksheet}
\subsection{Data}

\subsubsection{DC Response Data}

Record the equivalent resistance and the capacitance of your circuit:

\begin{center}
$R_{eq}=$~ \rule{3cm}{.1mm}~~~~
$C$=~ \rule{3cm}{.1mm} \\
\end{center}

\noindent
Enter your voltage-time pairs in the table below.

\noindent
{\bf Take Note:} Your first measurement should be of the voltage at the beginning of
the discharging phase. (That's the peak of the curve on the screen.) 

\begin{table}[htb]
\begin{center}
\begin{tabular}{|c|c|c|c|c|}
\hline
\multicolumn{5}{|c|}{Voltage and Time Coordinates} \\
\hline
Voltage & Time & & Voltage & Time \\
\hline
\hspace*{3cm} & \hspace*{3cm} & \hspace*{.1cm} & \hspace*{3cm} & \hspace*{3cm} \\
& & & & \\
\hline
& & & & \\
& & & & \\
\hline
& & & & \\
& & & & \\
\hline
& & & & \\
& & & & \\
\hline
& & & & \\
& & & & \\
\hline

\end{tabular}
\end{center}
\caption{V vs. t measurements for discharging capacitor.}
\label{tab:RC:decay}
\end{table}

\subsubsection{AC Response Data}

Calculate (without uncertainty) the {\it natural} cutoff frequency of your circuit.
Record this value:

\begin{center}
f=~ \rule{3cm}{.1mm}
\end{center}

\noindent
Enter your amplitude-period pairs in the table below. 

\noindent
{\bf Take Note:} Your first amplitude measurement should be taken from a signal with
a frequency of around 100 Hz. Once you have taken this measurement, {\bf DO NOT adjust
the AMPLITUDE KNOB on the function generator.} 

\noindent
{\bf Take Note:} A good majority (at least seven) of your amplitude 
measurements should be taken from signals with frequencies {\it above} the natural 
cutoff frequency of the circuit.

\begin{table}[htb]
\begin{center}
\begin{tabular}{|c|c|c|c|c|}
\hline
\multicolumn{5}{|c|}{Amplitude vs. Period Measurements} \\
\hline
Amplitude & Period & & Amplitude & Period \\
\hline
\hspace*{3cm} & \hspace*{3cm} & \hspace*{.1cm} & \hspace*{3cm} & \hspace*{3cm} \\
& & & & \\
\hline
& & & & \\
& & & & \\
\hline
& & & & \\
& & & & \\
\hline
& & & & \\
& & & & \\
\hline
& & & & \\
& & & & \\
\hline

\end{tabular}
\end{center}
\caption{Amplitude vs. Period Measurements.}
\label{tab:RC:lowpass}
\end{table}

\subsection{Calculations}

\subsubsection{DC Response}

From your data in Table~\ref{tab:RC:decay}, plot ln$(V/V_0)$ versus $t$, with error
bars. ($V_0$ is the voltage of the signal at the beginning of the decay curve.) Find
the weighted best-fit slope and intercept of this plot with uncertainties,
and record them below.

\begin{center}
Slope=~ \rule{3cm}{.1mm} ~~~~
Intercept=~ \rule{3cm}{.1mm}
\end{center} 
\vspace*{.5cm}

\noindent
From equation (4.4) in the lab manual, determine how the slope of your plot should be
related to the time constant of your circuit. With this information, 
calculate the time constant of your circuit, 
$\tau _{meas}$, from the slope of your plot. Now, from the nominal values of
resistance and capacitance of your circuit, calculate the expected time constant,
$\tau _{nom}$. Record these below. Be sure to include units and uncertainties.

\begin{center}
$\tau _{meas}=$~ \rule{3cm}{.1mm} ~~~~
$\tau _{nom}=$~ \rule{3cm}{.1mm}
\end{center}
\vspace*{.5cm}

\noindent
From $\tau _{meas}$, and the nominal resistance of your circuit, $R_{eq}$, calculate
a value for the capacitance of your circuit, $C_{meas}$. Record this below. Be sure to
include units and uncertainties.

\begin{center}
$C_{meas}=$~ \rule{3cm}{.1mm}
\end{center}

\noindent
{\it Sample Calculations:}
\newpage

\subsubsection{AC Response}

From your data in Table~\ref{tab:RC:lowpass}, make a plot of amplitude versus {\bf
natural frequency}, with error bars. 
\vspace*{.5cm}

\noindent
Now, using the value you {\bf measured} for the time constant of the circuit, make a
plot of the {\it theoretical} ampltiude versus natural frequency, as predicted by
equation (4.7) in the lab manual. In other words, plot this expression,

$$
V_0 \over {\sqrt {1+(2\pi f \tau_{meas})^2} },
$$

\noindent
versus frequency on the same plot as your amplitude versus frequency plot. In this
case, $V_0$ is the amplitude you measured for the signal with frequency around 100 Hz,
$\tau_{meas}$ is the time constant you measured for your circuit, and $f$ is frequency.
{\bf Hint:} plot this at the same frequencies at which you measured your amplitude 
values. Make sure both plots contain error bars.

\vspace*{.2cm}
\noindent
{\it Sample Calulations:}
\newpage

\subsection{Discussion}
\subsubsection{DC Response}
Explain how equation (4.4) in the lab manual leads you to expect a plot of ln($V/V_0$)
versus $t$ to be linear.
\vspace*{2cm}

\noindent
Was your plot of ln($V/V_0$) versus $t$ linear?
\vspace*{.3cm}

\noindent
What do you expect the slope and intercept of this plot to be?
Write your answers in terms of $\tau _{meas}$, the nominal time constant of the 
circuit.
\vspace*{1.4cm}

\noindent 
Compare the value of the intercept of your plot to its expected value.
\vspace*{1.4cm}

\noindent
Compare the value of the circuit's time constant that you obtained from the plot ($\tau
_{meas}$), to its expected value ($\tau _{nom}$).
\vspace*{1.4cm}

\noindent
Compare the value you measured for the capacitance of your circuit ($C_{meas}$), to
its nominal value ($C_{nom}$).
\vspace*{1.4cm}

\noindent 
Discuss the conclusions which can be drawn from the results of the above three
comparisons.
\newpage 

\subsubsection{AC Response}  

Compare your plot of measured amplitude versus frequency to your plot of theoretical
amplitude versus frequency. In particular, discuss the degree to which they agree
within uncertainties.
\vspace*{2cm}

\noindent
Discuss what conclusions can be drawn on the basis of this comparison.
\vspace*{3.5cm}

\subsection{Conclusion}

Write a {\it brief} (that is, a one or two paragraph) conclusion for this lab (on your
own paper). In it, you should summarize the physical principles which were meant to be
illustrated in this experiment. You should also describe the degree to which your data
supported these principles.


% Go back to ordinary section numbering
\renewcommand{\thesection}{\thechapter.\arabic{section}}



