\section{DC Circuits Worksheet}

\subsection{Batteries in Series}

\subsubsection{Procedure}

Measure the voltage supplied by each of the batteries (enter your results into 
Table~\ref{tab:DC:battseries}).  Take care to distinguish between the 
batteries once you've made your measurements. Connect the batteries as in 
Figure~\ref{fig:DC:procbatt}a and measure the voltage across the combination.  
\begin{figure}[htb]
\centerline{\epsfxsize=10cm \epsfbox{procbatt.eps}}
\caption{Batteries in Series.}
\label{fig:DC:procbatt}
\end{figure}
Now connect the batteries in opposition, as in Figure~\ref{fig:DC:procbatt}b,
and measure the voltage.

\subsubsection{Data, Analysis and Results}
Calculate the voltages you expect to measure for the batteries in series and in
opposition, and enter these in Table~\ref{tab:DC:battseries}). Be sure to present
sample calculations (including uncertainty calculations) in the space provided
beneath the data table.
\newpage

\begin{table}[h]
\begin{center}
\begin{tabular}{|c|c|}
\hline
Voltage (battery 1) & Voltage (battery 2) \\ 
\hline
\hspace*{5cm} & \hspace*{5cm}\\
& \\
\hline
\hline
Voltage in Series & Expected Value \\
\hline
& \\
& \\
\hline
Voltage in Oppostition & Expected Value\\
\hline
& \\
& \\
\hline
\end{tabular}
\end{center}
\caption{Voltage measurements.}
\label{tab:DC:battseries}
\end{table}

{\it Sample Calculations:}

\vspace*{2cm}


\subsection{Resistance Measurement}
\subsubsection{Procedure}

Pick two resistors and use the color code to determine their nominal resistance
and tolerance. Use the ohmmeter setting of the multimeter to measure their
resistance. Enter these values into Table~\ref{tab:DC:resistmeas}.

We can also measure the resistance in the following manner, outlined
in Figure~\ref{fig:DC:procresist}.  
\begin{figure}
\centerline{\epsfxsize=6cm \epsfbox{procresist.eps}}
\caption{Circuit used to measure resistance.}
\label{fig:DC:procresist}
\end{figure}
Begin by connecting the power source to 
the breadboard. Set the power source to a voltage of approximately 3 V. Pick 
one of your resistors and build the circuit in Figure~\ref{fig:DC:procresist}. 
You will need to include jumper wires so that you can make the necessary 
connection for the current measurement.  Measure the voltage across and the 
current through the resistor.  Repeat the measurement with your other resistor.  

\subsubsection{Data, Analysis, and Results}
Use Ohm's law to calculate the resistance with uncertainty for each resistor
using your voltage and current measurements. Enter these in
Table~\ref{tab:DC:resistmeas}. 
Calculate the average resistance of each resistor using 
equation (0.1) in the lab manual and the standard deviation with 
equation (0.2) in the lab manual. \\  

\noindent
Present all pertinent sample calculations.
\indent
\newpage

\begin{table}[htb]
\begin{center}
\begin{tabular}{|c|c|c|}
\hline
\multicolumn{3}{|c|}{Resistor 1} \\
\hline 
Voltage & Current & Ohm's Law Resistance \\ 
\hline
\hspace*{3cm} & \hspace*{3cm} & \hspace*{3cm} \\ 
& &  \\ 
\hline
Color Code & Ohmmeter & Average Resistance \\ 
\hline
& &  \\
& &  \\
\hline
\hline
\multicolumn{3}{|c|}{Resistor 2} \\
\hline 

Voltage & Current & Ohm's Law Resistance \\
\hline
\hspace*{3cm} & \hspace*{3cm} & \hspace*{3cm} \\ 
& &  \\ 
\hline

Color Code & Ohmmeter & Average Resistance \\ 
\hline
& &  \\
& &  \\
\hline
\end{tabular}
\end{center}
\caption{Resistance measurements.}
\label{tab:DC:resistmeas}
\end{table}

{\it Sample Calculations:}
\vspace*{3.5cm}



\newpage

\subsection{Resistors in Series and Parallel}
\subsubsection{Procedure}

Using the same resistors as above, build the series and parallel circuits in
Figure~\ref{fig:DC:procserpar}.  
\begin{figure}[htb]
\centerline{\epsfxsize=14cm \epsfbox{procserpar.eps}}
\caption{Series and parallel resistance measurements.}
\label{fig:DC:procserpar}
\end{figure}
Do the following procedure for {\it each} circuit. 

Measure the 
current and voltage necessary to calculate the resistance of each resistor 
combination. Also, devise a method of using the {\it ohmmeter} to measure the 
equivalent resistance. (Hint: It might be more convenient to use different 
connections on the breadboard.) Enter your measurments in
Table~\ref{tab:DC:measserpar}.  

\subsubsection{Data, Analysis, and Results}
Calculate the average and standard deviation of your two measurements for each
circuit, and enter these values in Table~\ref{tab:DC:measserpar}. 

\noindent
Now, calculate
values for the equivalent resistance of each circuit using the color codes of the
resistors. Do the same using the other individual values of resistance you
obtained in the previous section. In other words, those you measured 
with the ohmmeter, and those you obtained from Ohm's Law. 
Calculate the average and standard deviation of these measurements.
Enter these values in Table~\ref{tab:DC:measserpar}. 
\indent
\newpage

\begin{table}[htb]
\begin{center}
\begin{tabular}{|c|c|c|}
\hline
\multicolumn{3}{|c|}{Series Resistors (Measurements)} \\
\hline 
Voltage & Current & Ohm's Law Resistance \\
\hline
\hspace*{3cm} & \hspace*{3cm} & \hspace*{3cm} \\ 
& &  \\ 
\hline
Ohmmeter & Average & \\ 
\hline
& &  \\
& &  \\
\hline
\hline

\multicolumn{3}{|c|}{Series Resistors (Calculated)} \\
\hline 
Color Codes & Ohmmeter & Ohm's Law Resistance \\
\hline
\hspace*{3cm} & \hspace*{3cm} & \hspace*{3cm} \\ 
& &  \\ 
\hline
Average &  & \\ 
\hline
& &  \\
& &  \\
\hline
\hline

\multicolumn{3}{|c|}{Parallel Resistors (Measurements)} \\
\hline 
Voltage & Current & Ohm's Law Resistance \\
\hline
\hspace*{3cm} & \hspace*{3cm} & \hspace*{3cm} \\ 
& &  \\ 
\hline
Ohmmeter & Average & \\
\hline
& &  \\
& &  \\
\hline
\hline

\multicolumn{3}{|c|}{Parallel Resistors (Calculated)} \\
\hline 
Color Codes & Ohmmeter & Ohm's Law Resistance \\
\hline
\hspace*{3cm} & \hspace*{3cm} & \hspace*{3cm} \\ 
& &  \\ 
\hline
Average &  & \\
\hline
& &  \\
& &  \\
\hline


\end{tabular}
\end{center}
\caption{Series and parallel resistance measurements.}
\label{tab:DC:measserpar}
\end{table}

\newpage

\noindent
{\it Sample Calculations:}
\vspace*{6cm}



\newpage

\subsection{Temperature Dependence of Resistance}
\subsubsection{Procedure}
To begin with, turn the voltage on your power supply to {\it zero}.
Then, devise a circuit to measure the resistance of a flashlight bulb. (Just 
model it on the circuits that we've been using so far.) Turn the voltage up to
about 0.5 V and measure the voltage across and the current through the bulb. 
Make approximately ten measurements of the voltage and current for widely 
spread voltage values in the region 0 V to approximately 3 V. Enter these values
in Table~\ref{tab:DC:ltbulbplot}.  

\begin{table}[htb]
\begin{center}
\begin{tabular}{|c|c|}
\hline
\multicolumn{2}{|c|}{Light bulb}\\
\hline
I & V \\
\hline
\hspace*{5cm} & \hspace*{5cm} \\
& \\
\hline
& \\
& \\
\hline
& \\
& \\
\hline
& \\
& \\
\hline
& \\
& \\
\hline
& \\
& \\
\hline
& \\
& \\
\hline
& \\
& \\
\hline
& \\
& \\
\hline
& \\
& \\
\hline
\end{tabular}
\end{center}
\caption{V versus I for a light bulb.}
\label{tab:DC:ltbulbplot}
\end{table}

\begin{table}[htb]
\begin{center}
\begin{tabular}{|c|c|c|}
\hline
\multicolumn{3}{|c|}{Resistor} \\
\hline 
Voltage & Current & Ohm's Law Resistance \\ 
\hline
\hspace*{3cm} & \hspace*{3cm} & \hspace*{3cm} \\ 
& &  \\ 
\hline
Ohmmeter &  &  \\ 
\hline
& &  \\
& &  \\
\hline
\end{tabular}
\end{center}
\caption{Resistance Measurements.}
\label{tab:DC:newres}
\end{table}

You don't want to burn out the bulb, so don't make the voltage too high, but 
be sure that the bulb is lit for several of the measurements.  \\

Now, choose a resistor.
Find the resistance of this resistor using both the Ohmmeter and Ohm's Law.
Enter these measurements in Table~\ref{tab:DC:newres}.
Now replace the bulb in this circuit by this resistor.  
Repeat the measurements of voltage and current
over the same range of voltage values you used for the bulb, and enter these
values in Table~\ref{tab:DC:resisplot}. 


\begin{table}[htb]
\begin{center}
\begin{tabular}{|c|c|}
\hline
\multicolumn{2}{|c|}{Resistor}\\
\hline
I & V \\
\hline
\hspace*{5cm} & \hspace*{5cm} \\
& \\
\hline
& \\
& \\
\hline
& \\
& \\
\hline
& \\
& \\
\hline
& \\
& \\
\hline
& \\
& \\
\hline
& \\
& \\
\hline
& \\
& \\
\hline
& \\
& \\
\hline
& \\
& \\
\hline
\end{tabular}
\end{center}
\caption{V versus I for a resistor.}
\label{tab:DC:resisplot}
\end{table}


\subsubsection{Data, Analysis, and Results}

Using the computer, make a plot of the voltage versus the current for both 
sets of measurements. Make a linear fit to the resistor graph and write
down the slope and intercept in Table~\ref{tab:DC:slopeinter}.

\begin{table}[htb]
\begin{center}
\begin{tabular}{|c|c|}
\hline
\multicolumn{2}{|c|}{Resistor} \\
\hline
Slope & Intercept \\
\hline
\hspace*{5cm} & \hspace*{5cm} \\
& \\
\hline
\end{tabular}
\end{center}
\caption{Slope and Intercept for the resistor $V$ vs.\ $I$ plot.}
\label{tab:DC:slopeinter}
\end{table}

\newpage

\newpage

\subsection{Internal Resistance of a Dry Cell} 
\subsubsection{Procedure}

Disconnect the power supply from the breadboard and connect one of the 
batteries to the breadboard power contacts. Build the extremely simple circuit
shown in Figure~\ref{fig:DC:procintresist}. 
\begin{figure}[htb]
\centerline{\epsfxsize=3cm \epsfbox{procintresist.eps}}
\caption{Internal resistance measurement.}
\label{fig:DC:procintresist}
\end{figure}
Make measurements of the voltage {\it across the battery} and the current
through the resistor for about ten different values of resistance. Use
different resistors and, if necessary, series and parallel combinations
of the resistors to make sure you get ten data points. It isn't necessary to
note the resistances you used; all we want to do is vary the load on the 
battery. 



\begin{table}[htb]
\begin{center}
\begin{tabular}{|c|c|}
\hline
\multicolumn{2}{|c|}{Dry cell}\\
\hline
I & V \\
\hline
\hspace*{5cm} & \hspace*{5cm} \\
& \\
\hline
& \\
& \\
\hline
& \\
& \\
\hline
& \\
& \\
\hline
& \\
& \\
\hline
& \\
& \\
\hline
& \\
& \\
\hline
& \\
& \\
\hline
& \\
& \\
\hline
& \\
& \\
\hline
\end{tabular}
\end{center}
\caption{V versus I for a dry cell.}
\label{tab:DC:intres}
\end{table}
\clearpage

\subsubsection{Data, Analysis, and Results}

Plot the voltage versus the current and obtain the slope and intercept of a
linear fit to the plot.  Enter these values in Table~\ref{tab:DC:battslope}.

\begin{table}[htb]
\begin{center}
\begin{tabular}{|c|c|}
\hline
\multicolumn{2}{|c|}{Dry cell} \\
\hline
Slope & Intercept \\
\hline
\hspace*{5cm} & \hspace*{5cm} \\
& \\
\hline
\end{tabular}
\end{center}
\caption{Slope and Intercept for the dry cell $V$ vs.\ $I$ plot.}
\label{tab:DC:battslope}
\end{table}







\subsection{Discussion}
\subsubsection{Batteries in a Series}
Compare your results with those that you would expect from your knowledge of 
voltage sources in series.  Do the values match within uncertainty? \\
\subsubsection{Resistance Measurements}

For each resistor, you will have four values of resistance: nominal, that
measured with the ohmmeter, that measured in the circuit, and the average. 
Do all these values agree within uncertainty for both resistors? 
  
\vspace*{1cm}
\noindent
Which of these values would you report as {\it the} resistance? Why?
\indent
\indent

\subsubsection{Resistors in Series and Parallel}
From the results for each circuit, answer the following question, and discuss
the conclusions which can be drawn from each of your answers. 
Does the measured resistance agree with the 
calculated equivalent resistance, within the standard deviations of the 
two results? 

\subsubsection{Temperature Dependence of Resistance}
\noindent
Are either of the graphs linear?\\ 
\ \\
\noindent
What does the slope at each point of the graphs measure? (Use Ohm's law to
determine this.) 
\vspace*{2cm}

\noindent
What does the graph for the bulb measurements tell you about the temperature
dependence of resistance? 
\vspace*{2cm}

\noindent
Use the slope you obtained from the linear fit to calculate another value
for the resistance of the resistor. \\
\vspace*{2mm}
$$R=\mbox{\hspace*{3cm}}$$
\vspace*{1mm}\\
With the two other measurements, you now
have three experimental values for the resistance of your resistor. 
What is the average and standard deviation of the three measurements? \\
\vspace*{1.5cm} \\
Compare the relative uncertainty in the average to those in the individual 
measurements. \\
\ \\


\subsubsection{Internal Resistance of a Dry Cell}
What do the slope and $V$-intercept of the graph measure? \\
\vspace*{1cm}\\
What is the internal resistance of the battery? \\
\vspace*{2mm}
$$r=\mbox{\hspace*{3cm}}$$

\subsection{Conclusion}
Write a {\it brief} conclusion for this lab (on your own paper). This 
conclusion should contain a summary of what physical principles were 
supposed to be demonstrated in these experiment and a {\it brief} description 
of the degree to which your data supported these principles.



% Go back to ordinary section numbering
\renewcommand{\thesection}{\thechapter.\arabic{section}}


