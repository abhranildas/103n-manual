\vfill
\pagebreak

%  Label worksheets by \thechapter.W
\renewcommand{\thesection}{\thechapter.W}


\section{Imaging Optics Worksheet}
\subsection{Data}

\subsubsection{Focal Length and Magnification of a Lens}

Place the lens {\bf HALF} its nominal focal length from the object slide. 
Record your answers to the following questions.
\vspace*{.5cm}

\noindent
When you look through the lens, can you see a {\it clear} image of the object? If
you can, is this image upright or inverted?
\vspace*{1.5cm}

\noindent
Are you able to project (through the lens) a {\it clear} image of the object on the 
notecard? If you are, is this image upright or inverted?
\vspace*{1.5cm}

\noindent
Place the lens {\bf TWICE} its nominal focal length from the object slide. 
Record your answers to the following questions.
\vspace*{.5cm}

\noindent
When you look through the lens, can you see a {\it clear} image of the object? If
you can, is this image upright or inverted?
\vspace*{1.5cm}

\noindent
Are you able to project (through the lens) a {\it clear} image of the object on the 
notecard? If you are, is this image upright or inverted?


\newpage

\noindent
Record your five image distance versus object distance measurements for each
lens in the tables below. For the first pair of measurements for each lens, also
measure the image size and the object size and record them in the designated
place. Be sure to include uncertainties.

\begin{table}[htb]
\begin{center}
\begin{tabular}{|c|c|c|c|}
\hline
\multicolumn{4}{|c|}{Measurements of $i$ vs. $o$ for ``136 mm'' lens.} \\
\hline
Image Distance ($i$) & Object Distance ($o$) & Image Size & Object Size \\
\hline
\hspace*{3cm} & \hspace*{3cm} & \hspace*{3cm} & \hspace*{3cm} \\
& & &  \\
\hline
& & NOT & NOT  \\
& & HERE& HERE \\
\hline
& & NOT & NOT \\
& & HERE & HERE\\
\hline
& & NOT & NOT \\
& & HERE & HERE \\
\hline
& & NOT & NOT \\
& & HERE & HERE \\
\hline
\end{tabular}
\end{center}
\caption{$i$ vs. $o$ measurements for ``136 mm'' lens.}
\label {tab:OP:136}
\end{table}


\begin{table}[htb]
\begin{center}
\begin{tabular}{|c|c|c|c|}
\hline
\multicolumn{4}{|c|}{Measurements of $i$ vs. $o$ for ``238 mm'' lens.} \\
\hline
Image Distance ($i$) & Object Distance ($o$) & Image Size & Object Size \\
\hline
\hspace*{3cm} & \hspace*{3cm} & \hspace*{3cm} & \hspace*{3cm} \\
& & &  \\
\hline
& & NOT & NOT  \\
& & HERE & HERE \\
\hline
& & NOT & NOT \\
& & HERE & HERE \\
\hline
& & NOT & NOT \\
& & HERE & HERE \\
\hline
& & NOT & NOT \\
& & HERE & HERE \\
\hline
\end{tabular}
\end{center}
\caption{$i$ vs. $o$ measurements for ``238 mm'' lens.}
\label {tab:OP:238}
\end{table}

\noindent
Now, plot $1/i$ versus $1/o$ for each lens. Obtain a weighted least-squares
linear fit for each plot. From these fits, determine the focal length of each lens.
Record these values below.

\begin{center}
$f_{136}=$~ \rule{3cm}{.1mm}~~~~
$f_{238}$=~ \rule{3cm}{.1mm} 
\end{center}

\newpage

\subsubsection{A Compound Lens System}

{\bf Note:} In the procedure of this section, you should use the {\it measured}
focal lengths of the lenses. In other words, {\bf DO NOT} use 136mm and 238mm
as the focal lengths of your lenses. Instead, use the values you have just
determined for $f_{136}$ and $f_{138}.$
\vspace*{.5cm}

\noindent
{\it Without uncertainty} for now,  calculate the position of the image of the first
lens, $i_{136},$ using the thin lens equation. Record this below.

\begin{center}
$i_{136}=$~\rule{3cm}{.1mm}
\end{center}
\vspace*{.5cm}

\noindent
Measure the image distance for the image of the second lens, $i_{comp}.$
Record this below with uncertainty.

\begin{center}
$i_{comp}=$~\rule{3cm}{.1mm}

\end{center}
\vspace*{.5cm}

\noindent
Measure the object size $s_o$ and the image size $s_i$ for the image of the 
second lens. Record these below with uncertainty.

\begin{center}
$s_i=$~\rule{3cm}{.1mm} ~~~~
$s_o=$~\rule{3cm}{.1mm}
\end{center}

\newpage

\subsection{Calculations}

\subsubsection{Focal Length and Magnification of a Lens}
From your plots, record the slopes and intercepts on the lines below (with
uncertainty).

\noindent
{\it ``136mm'' lens:}
\begin{center}
Slope=~\rule{3cm}{.1mm} ~~~~
Intercept=~\rule{3cm}{.1mm}
\end{center}
\vspace*{.5cm}

\noindent
{\it ``238mm'' lens:}
\begin{center}
Slope=~\rule{3cm}{.1mm} ~~~~
Intercept=~\rule{3cm}{.1mm}
\end{center}
\vspace*{.5cm}

\noindent
{\it With uncertainties} this time, calculate the measured focal lengths of
each lens, and record them below.

\begin{center}
$f_{136}=$~\rule{3cm}{.1mm} ~~~~
$f_{238}=$~\rule{3cm}{.1mm}
\end{center}
\noindent
{\it Sample Calculations:}
\vspace*{2cm}

\noindent
From your image and object sizes, calculate the magnification you measured for
each of the lenses, $m_{136}$ and $m_{238}.$ Record these below with
uncertainties.

\begin{center}
$m_{136}=$~\rule{3cm}{.1mm} ~~~~
$m_{238}=$~\rule{3cm}{.1mm}
\end{center}
\vspace*{.5cm}

Now, from the image and object distances in the cases where you measured the
image and object sizes, calculate the {\it predicted} magnification of each lens,
$m'_{136}$ and $m'_{238}.$ Record these below with uncertainties.

\begin{center}
$m'_{136}=$~\rule{3cm}{.1mm} ~~~~
$m'_{238}=$~\rule{3cm}{.1mm}
\end{center}
{\it Sample Calculations:}
\newpage

\subsubsection{A Compound Lens System}
With uncertainties, and using the measured focal lengths, 
calculate the image distance, $i'_{comp}$ predicted by the thin lens
equation for the compound lens system.  Record this value below.

\begin{center}
$i'_{comp}=$~\rule{3cm}{.1mm}
\end{center}
\vspace*{.5cm}

\noindent
Using the image and object sizes you measured, calculate the magnification 
you measured for this image, $m_{comp}.$ Also, using image and object
distances for both lenses, predict what the magnification of the compound
system should be. Call this $m'_{comp}.$ 
Record these values below (with uncertainties).

\begin{center}
$m_{comp}=$~\rule{3cm}{.1mm} ~~~~
$m'_{comp}=$~\rule{3cm}{.1mm}
\end{center}
{\it Sample Calculations:}
\newpage

\subsection{Discussion}
\subsubsection{Focal Length and Magnification of a Lens}
Consider the image(s) you were able to observe when the lens was half its
nominal focal length from the object slide. Were they real or virtual?
\vspace*{.3cm}

\noindent
Consider the image(s) you were able to observe when the lens was twice its
nominal focal length from the object slide. Were they real or virtual?
\vspace*{.3cm}

\noindent
Discuss the degree to which your observations were consistent with the
predictions made by the ray diagrams on p.114 of the lab manual.
\vspace*{2cm}

\noindent
Compare the slopes of each of your plots to the slope predicted by the thin lens
equation.
\vspace*{1.4cm}

\noindent 
Compare the nominal focal lengths of each of your lenses to the measured focal
length of each of your lenses 
\vspace*{1.4cm}

\noindent
Compare the measured magnifications of each of your lenses to the predicted
magnifications of each of your lenses. 
\vspace*{1.4cm}

\noindent
Is the sign convention for magnification consistent with your observations?
\vspace*{.3cm}

\noindent 
Discuss the conclusions which can be drawn from the results of the above 
comparisons.
\newpage 

\subsubsection{A Compound Lens System}  

Is the image of the first lens alone real or virtual?
\vspace*{.3cm}

\noindent
Compare the image distance you measured for the compound lens system,
$i_{comp},$ to the predicted image distance $i'_{comp}.$
\vspace*{1.4cm}

\noindent
Compare the magnification you measured for the compound lens system,
$m_{comp},$ with the predicted magnification of the compound lens system,
$m'_{comp}.$
\vspace*{1.4cm}

\noindent
Discuss the conclusions that can be drawn from the above comparsions.
\vspace*{3cm}

\subsection{Conclusion}

Write a {\it brief} (that is, a one or two paragraph) conclusion for this lab (on your
own paper). In it, you should summarize the physical principles which were meant to be
illustrated in this experiment. You should also describe the degree to which your data
supported these principles.





% Go back to ordinary section numbering
\renewcommand{\thesection}{\thechapter.\arabic{section}}


