\section{Introduction}

The generators that supply electricity to our homes and the motors in our cars
are important devices whose mechanism of operation is typically taken for 
granted. Today, we'll explore the fundamental physical law responsible for 
their operation, Faraday's law of induction.

Faraday's law states that a changing magnetic field can produce an induced 
voltage in an electric circuit.  The device we will study is a simple
transformer, two coils of wire linked by an iron core.  It operates by
running an alternating current (the input) through one coil (the primary coil)
to generate an alternating magnetic field in the iron. (This phenomenon is 
described by another fundamental law, called Amp\`{e}re's law.)  The magnetic 
field passes through the second coil (the secondary coil), where, by Faraday's 
law, it induces an alternating potential (the output) across the coil. This 
voltage can be viewed on the oscilloscope (remember that it's varying in time),
so that its properties can be studied.

In particular, we will examine the output signal from the transformer when we
use a sinusoidal input. We will see that the output is also sinusoidal and we
will examine the dependence of the output amplitude on both the frequency of
the input signal and the number of turns in the secondary coil. We will also
examine the output for other input signals: triangle waves and square waves.
Examining these signals will reveal that the output signals is proportional
to the {\it time derivative} of the input signal.

\vfill
\pagebreak

\section{Theory}

\subsection{References}

Faraday's law of induction is introduced in Serway, Chapter~31 (Faraday's
Law), p.~874.  Chapter~33 (Alternating Current Circuits), p.~927 is also 
useful, since we're using AC circuits.  The magnetic field of a solenoid is 
discussed in Sections~30.4 (The Magnetic Field of a Solenoid), p.~845 and~30.5
(The Magnetic Field Along the Axis of a Solenoid),p.~847.

\subsection{Magnetic Flux}

Whenever we have a vector field present we may speak of the flux of that field 
through a surface.  If we have a magnetic field $\vec{B}$ which passes through 
an open surface $S$ as in Figure~\ref{fig:ind:fluxdef},
\begin{figure}[htb]
\centerline{\epsfxsize=7cm \epsfbox{induction_2/fluxdef.eps}}
\caption{A magnetic flux $\Phi_M$ passes through the surface $S$.}
\label{fig:ind:fluxdef}
\end{figure}
then we define the magnetic flux to be the integral
$$
\Phi_M=\int_S \vec{B}\cdot d\vec{A},
$$
where $d\vec{A}$ is the perpendicularly oriented differential area element.

Let's calculate the magnetic flux which passes through a long thin solenoid, 
of $N$ turns, length $L$, radius $a\ll L$, and carrying a current $I$, 
illustrated in Figure~\ref{fig:ind:solenoid}. 
\begin{figure}[htb]
\centerline{\epsfxsize=9cm \epsfbox{induction_2/solenoid.eps}}
\caption{A long thin solenoid (not to scale).}
\label{fig:ind:solenoid}
\end{figure}
We'll consider the field responsible for the flux to be that of the solenoid 
itself, so this is an example of {\it self-induction}. The magnetic field 
along the cross section of the solenoid is roughly constant and given by
$$ B= \frac{\mu_0 N I}{L}; $$
for a more complete discussion, see Serway. The flux is then
\begin{eqnarray}
\Phi_M &=& \int_S \vec{B}\cdot d\vec{A} \nonumber \\
&=& \int_0^{2\pi} d\theta \int_0^a r dr \frac{\mu_0 NI}{L} \nonumber \\
&=& \frac{\mu_0 NI}{L} (\pi a^2) \nonumber \\
\Phi_M &=& \frac{\pi\mu_0N Ia^2}{L}.
\end{eqnarray}

\subsection{Faraday's Law}

Consider again the solenoid in Figure~\ref{fig:ind:solenoid}, but this time 
imagine that the magnetic field is time-varying and is coming from some
external source, perhaps another coil, rather than from a current running 
through the solenoid.  Faraday's law states that the magnetic flux through the 
solenoid induces a voltage
\begin{equation}
V=-N\frac{d\Phi_M}{dt} \label{eq:ind:faraday}
\end{equation}
across the solenoid.  The factor of $N$ appears because the flux $\Phi_M$ links
each of the $N$ coils of the solenoid; the voltages induced by $\Phi_M$ in 
each turn of the coil add in series. 

\subsection{The Transformer}

Let's now apply what we've learned to a transformer, illustrated in 
Figure~\ref{fig:ind:transformer}. 
\begin{figure}[htb]
\centerline{\epsfxsize=10cm \epsfbox{induction_2/transformer.eps}}
\caption{A simple transformer.}
\label{fig:ind:transformer}
\end{figure}
A voltage $V_{\mbox{in}}$ is supplied to a primary coil of $N_1$ turns,
producing a current $I_{\mbox{in}}$, which, from~(\ref{eq:ind:fluxgen}),
generates a flux 
$$\Phi= \frac{\pi\mu  N_1 a_1^2}{L_1} I_{\mbox{in}},
$$
in the iron core. We assume that, since the iron is a magnetic material, it
holds this flux and carries it through the secondary coil of $N_2$ turns;
since we aren't dealing with a field in vacuo anymore, we have replaced 
$\mu_0$ by $\mu$, the permeability of iron. 
Faraday's law~(\ref{eq:ind:faraday}) implies that a voltage, given by 
$$V_{\mbox{out}} = - N_2\frac{d\Phi}{dt},$$  
is generated across the secondary coil.  We can substitute in for $\Phi$ to 
write
$$
V_{\mbox{out}} = - \left( \frac{\pi\mu  N_1 a_1^2}{R L_1} \right) N_2 
\frac{d V_{\mbox{in}}}{dt}.
$$
Now, in our experimental setup, we will vary $N_2$ and $V_{\mbox{in}}$, but 
will not vary or measure $a_1,L_1,R$, or $N_1$. It will therefore be useful to 
define the (positive) constant $C=\pi\mu  N_1 a_1^2/RL_1$ and write
\begin{equation}
V_{\mbox{out}}= -C N_2 \frac{d V_{\mbox{in}}}{dt}, \label{eq:ind:transformer}
\end{equation}
which illustrates the dependencies that we will study, leaving the constant $C$ 
to conveniently sum up the (unknown, but unchanging) properties of the 
transformer.  We see that $V_{\mbox{out}}$ depends linearly on $N_2$, the 
number of turns in the secondary coil, and also depends linearly on the
time derivative of $V_{\mbox{in}}$.  We also note that there is an important
minus sign in~(\ref{eq:ind:transformer}) which corresponds to the minus sign
in Faraday's law~(\ref{eq:ind:faraday}). This is called Lenz's law: the induced
voltage {\it opposes} the change in flux.  
  
Let's see what the relationship~(\ref{eq:ind:transformer}) predicts for the 
output when we use a sinusoidal input
$$
V_{\mbox{in}} = A_{\mbox{in}} \sin\omega t.
$$
Using (\ref{eq:ind:transformer}) yields
$$
V_{\mbox{out}}= -C N_2 \omega A_{\mbox{in}} \cos\omega t.
$$
Figure~{\ref{fig:ind:sinusoid} shows a plot of a sine wave versus the negative
of a cosine wave.
There is therefore a phase difference of $90^\circ$ (that between sine and 
--cosine) between the input and output signals and the amplitudes of the input 
and output are related by 
$$ A_{\mbox{out}} = C N_2 \omega A_{\mbox{in}}. $$
$A_{\mbox{out}}$ depends linearly on both $N_2$ and $\omega$.
\begin{figure}[htb]
\centerline{\epsfxsize=10cm \epsfbox{induction_2/sinusoid.eps}}
\caption{A plot of a sine wave and a --cosine wave.}
\label{fig:ind:sinusoid}
\end{figure}

\section{Apparatus}

The apparatus we will use centers around the transformer box, illustrated in
Figure~\ref{fig:ind:connect}, which houses a transformer such as that in 
Figure~\ref{fig:ind:transformer}, but which allows us to vary the number of 
turns in the secondary coil. 
Note that each setting of the knob corresponds to
{\it four} turns in the coil}. We will use the function generator to provide 
our input signals and the oscilloscope to view both the input and output 
signals.


\vfill
\pagebreak

%  Label worksheets by \thechapter.W
\renewcommand{\thesection}{\thechapter.W}

\section{Electromagnetic Induction Worksheet}

{\bf \Large Name:}~ \rule{5cm}{.1mm}~~~~~~~
{\bf \Large Day/Time:}~\rule{3cm}{.1mm}\\

\subsection{In-Lab Procedure}
\label{sec:ind:proc}

Before doing anything else, examine the transformer box.  Turn it over so that
you can see the coils inside.  Does it resemble a transformer such as that 
we've been talking about, {\it i.e.}, that in 
Figure~\ref{fig:ind:transformer}?  Connect the box to the function generator
and oscilloscope as in Figure~\ref{fig:ind:connect}.
\begin{figure}[htb]
\centerline{\epsfxsize=12cm \epsfbox{induction_2/connect.eps}}
\caption{How to connect the transformer box.}
\label{fig:ind:connect}
\end{figure}

\subsubsection{Sinusoidal Input}

Set the function generator to produce a sinusoidal signal. Adjust the 
oscilloscope so that it displays both the input and output signals.  Make
sure that you adjust the channel 1 and channel 2 V/div settings so that the 
waveforms fill the screen.  Sketch the two waveforms on the gird below.
Sketch both the input and output on the same grid; use the second grid only if
you make a mistake.  {\bf Draw carefully.}  You will be answering questions 
based on this sketch in teh classroom.\\
\ \\
% Use the same oscilloscope grids as in the Measurements with the Oscilloscope
% lab.
\begin{tabular}{ccc}
\epsfxsize=7cm \epsfbox{scope_2/scope.eps} & \hspace{1cm} &
\epsfxsize=7cm \epsfbox{scope_2/scope.eps}
\end{tabular}\\

\noindent Measure the phase shift, %\Delta t$, between the 2 waveforms, and their periods..
$$\Delta t = $$.  \\
$$\tau1 = 



Measure the output {\it amplitude}, $A_{\mbox{out}}$, versus the frequency, 
$\omega$ of the input for at least 10 data points.  Make sure that you 
calculate the frequency values by measuring the period of the input wave on 
the oscilloscope; {\it do not} trust the values given by the function 
generator. Plot $A_{\mbox{out}}$ versus $\omega$ with a few representative 
error bars and comment on the linearity of the graph. Is your data in 
agreement with Faraday's law?

Now measure the output amplitude versus the number of turns in the secondary
coil, $N_2$, for each setting that the knob allows.  Plot $A_{\mbox{out}}$ 
versus $N_2$ with a few error bars and comment on your data's agreement with 
Faraday's law.

\subsection{Triangle and Square Wave Input}

Set the function generator to produce a triangle wave and adjust the 
oscilloscope to display the input and output signals. Sketch the input and
output signals. Using the oscilloscope (not just your sketch), carefully 
examine the input signal.  How close does it come to being a triangle wave?
Are the peaks perfectly sharp or do they round off gently? You might want to
adjust the oscilloscope to zoom into the peaks to answer this. As you did with
the sinusoidal input, compare the output signal with the negative time 
derivative of the input. You should find it helpful to answer the same 
questions. Does the output signal agree with Faraday's law?

Now set the function generator to produce a square wave and make a sketch of
the input and output signals, as before.  Does the output signal resemble a 
smooth curve?  Again, carefully examine the input signal, re-adjusting the 
oscilloscope whenever necessary. How good a square wave do you have? Are the
corners of the waves perfectly sharp? Do the sides of the square wave curve
or have a slope? As with your other sketches, compare the output signal with 
the negative time derivative of the input. Does the output signal agree with 
Faraday's law?

\vfill
\pagebreak

%  Label worksheets by \thechapter.W
\renewcommand{\thesection}{\thechapter.W}

\section{Electromagnetic Induction Worksheet}

{\bf \Large Name:}~ \rule{5cm}{.1mm}~~~~~~~
{\bf \Large Day/Time:}~\rule{3cm}{.1mm}\\
\ \\
{\bf Instructions:} This worksheet is written specifically for using the 
computers in the laboratory to draw your graphs.  Refer 
to Section~\ref{sec:ind:proc} (Procedure and Analysis) for more information.  

\subsection{Sinusoidal Input}

Use the grid for your sketch of the input and output signals (remember to 
sketch both the input and output on the same grid; use the second grid only if
you make a mistake):\\
\ \\
% Use the same oscilloscope grids as in the Measurements with the Oscilloscope
% lab.
\begin{tabular}{ccc}
\epsfxsize=7cm \epsfbox{scope_2/scope.eps} & \hspace{1cm} &
\epsfxsize=7cm \epsfbox{scope_2/scope.eps}
\end{tabular}\\
\ \\

\subsection{Triangle and Square Wave Input}

Sketch the input triangle wave and the output on the grid:\\ 
\ \\
\begin{tabular}{ccc}
\epsfxsize=7cm \epsfbox{scope_2/scope.eps} & \hspace{1cm} &
\epsfxsize=7cm \epsfbox{scope_2/scope.eps}
\end{tabular}\\
\ \\

\ \\
Sketch the input square wave and the output:\\
\ \\
\begin{tabular}{ccc}
\epsfxsize=7cm \epsfbox{scope_2/scope.eps} & \hspace{1cm} &
\epsfxsize=7cm \epsfbox{scope_2/scope.eps}
\end{tabular}\\
\ \\

 

\subsection{Calculations $\&$ Discussions}
\subsubsection{Sinusoidal Input}
Calculate the phase difference between the input and output signals with an 
uncertainty. Show your work.\\ 
\vspace*{4cm}\\
\hspace*{2cm} {$\phi$ =~\rule{3cm}{.1mm}}\\ 

Attach your data, graphs, and discussion on separate paper.  Be sure that you
answer all of the questions in the lab manual.
\subsubsection{Triangle and Square Wave Input}
Add your discussion of this graph to the discussion section you made for the 
sinusoidal input. Again, be sure that you answer all of the questions in
the manual.\\
Discuss this sketch in the discussion section you formed above.



Does the output resemble the negative of the time derivative of the input? 
Explain your answer in detail by answering the following questions.  Does the 
output reach a maximum when the time rate of change of the input is a minimum? 
Is it a minimum when the derivative of the input is a maximum?  What happens 
when the slope of the input is zero? Are there any other characteristics of 
your sketch that help you make an interpretation? Is all of this consistent 
with Faraday's law, particularly the minus sign that appears?

Measure the angular phase difference between the input and output signals.  
Does it agree within uncertainty with the phase difference predicted by 
Faraday's law?  

% Go back to ordinary section numbering
\renewcommand{\thesection}{\thechapter.\arabic{section}}
