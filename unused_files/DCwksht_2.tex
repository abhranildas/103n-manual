\newpage
%  Label worksheets by \thechapter.W
\renewcommand{\thesection}{\thechapter.W}

\section{DC Circuits Worksheet Part I}

{\bf \Large Name:}~ \rule{5cm}{.1mm}~~~~~~~
{\bf \Large Day/Time:}~\rule{3cm}{.1mm}\\


\noindent Open your lab manual to page 23(???) and follow the procedure 
listed there.  This worksheet is meant to help organize the procedure and your 
answers.  You need the manual to complete the lab correctly. 

\subsection{Procedure}

\subsubsection{Batteries in Series and Opposition}

\noindent Before starting, do these two simple calculations using your two batteries' voltage values. Calculate the expected voltage 
value you would read on a voltmeter first if they were in series 
and second if they were in opposition.  Space is provided here for the 
calculation.
\vskip\baselineskip
\vskip\baselineskip
\vskip\baselineskip
\vskip\baselineskip
\vskip\baselineskip 
\noindent Enter this value in the space titled ``Expected Value'' into 
Table~\ref{tab:DC:battseries}.
{\bf Following section 1.4.1 and Figure~\ref{fig:DC:procbatt}}, 
enter the measured voltages and uncertainties for the batteries in 
series and in opposition
into Table~\ref{tab:DC:battseries}.  Were you careful to 
distinguish between the 
batteries once you had made your measurements? 

\begin{table}[h]
\begin{center}
\begin{tabular}{|c|c|}
\hline
Battery 1 Voltage & Battery 2 Voltage  \\ 
\hline
\hspace*{5cm} & \hspace*{5cm}\\
& \\
\hline
\hline
Voltage in Series & Expected Value \\
\hline
& \\
& \\
\hline
Voltage in Opposition & Expected Value\\
\hline
& \\
& \\
\hline
\end{tabular}
\end{center}
\caption{Voltage measurements.}
\label{tab:DC:battseries}
\end{table}
\pagebreak

\subsubsection{Resistance Measurement}
\label{sec:DC:resist}

{\bf Following section 1.4.2 in you lab manual} enter the appropriate 
resistance measurements.  Note that you are taking {\bf 3} measurements of
{\bf 2} resistors.  First you use the color code, next the ohmmeter, and 
finally the circuit.  Using the circuit means you will make 2 measurements:
you will measure voltage and current across the resistor.  When you measure
{\bf current, place the ammeter IN SERIES} as dictates 
Figure~\ref{fig:DC:procresist}.   
Enter the color code resistance value (not the colors), the ohmmeter reading,
and the voltage and current readings with uncertainties from the circuit into 
Table~\ref{tab:DC:resistmeas}.
\begin{table}[htb]
\begin{center}
\begin{tabular}{|c|c|c|}
\hline
\multicolumn{3}{|c|}{Resistor 1} \\
\hline 
Color Code & Ohmmeter & Leave Blank\\ 
\hline
\hspace*{3cm} & \hspace*{3cm} & \hspace*{3cm} \\ 
& &  \\ 
\hline
Voltage & Current & Ohm's Law Resistance \\
\hline
& &  \\
& &  \\
\hline
\hline
\multicolumn{3}{|c|}{Resistor 2} \\
\hline 
Color Code & Ohmmeter & Leave Blank\\ 
\hline
\hspace*{3cm} & \hspace*{3cm} & \hspace*{3cm} \\ 
& &  \\ 
\hline
Voltage & Current & Ohm's Law Resistance \\
\hline
& &  \\
& &  \\
\hline
\end{tabular}
\end{center}
\caption{Resistance measurements.}
\label{tab:DC:resistmeas}
\end{table}


\subsubsection{Resistors in Series and Parallel}

{\bf Following section 1.4.3 in the lab manual} enter the appropriate
data into Table~\ref{tab:DC:SeriesParallelRmeas}.  Note that you now have 
chosen {\bf 2} resistors
and measured their resistances in {\bf 3} ways. In this section, you will
use those 2 resistors to build {\bf 2} circuits, one in series and one in
parallel.  You are doing {\bf 2 circuits} and will need to collect data 
for each circuit.
Using the values of ohmmeter and color codes from
Table~\ref{tab:DC:resistmeas}, 
determine the new circuits color code and ohmmeter resistance values.
({\it Note that the 2 resistors are in series or parallel and you will 
need to add the resistors appropriately.  })     
Enter the new nominal and ohmmeter measurements for the 2 
circuits into Table~\ref{tab:DC:SeriesParallelRmeas}.  Measure the voltage and
current for the 2 circuits once again being careful to place the {\bf Ammeter
in SERIES} with the resistors. 

\begin{table}[htb]
\begin{center}
\begin{tabular}{|c|c|c|}
\hline
\multicolumn{3}{|c|}{Series Resistors} \\
\hline 
Color Code & Ohmmeter & \\ 
\hline
\hspace*{3cm} & \hspace*{3cm} & \hspace*{3cm} \\ 
& &  \\ 
\hline
Voltage & Current & Ohm's Law Resistance \\
\hline
& &  \\
& &  \\
\hline
\hline
\multicolumn{3}{|c|}{Parallel Resistors} \\
\hline 
Color Code & Ohmmeter & \\ 
\hline
\hspace*{3cm} & \hspace*{3cm} & \hspace*{3cm} \\ 
& &  \\ 
\hline
Voltage & Current & Ohm's Law Resistance \\
\hline
& &  \\
& &  \\
\hline
\end{tabular}
\end{center}
\caption{Series and parallel resistance measurements.}
\label{tab:DC:SeriesParallelRmeas}
\end{table}

\subsection{Calculations \& Analysis}

\noindent
\subsubsection{Batteries in Series and Opposition}

\noindent Using Table~\ref{tab:DC:battseries}, compare your measured results 
with those that you calculated as expected.
Do the values match within uncertainty?  This last calculation is the first of
many comparisons.  You should do this mathematically by obtaining a 
percent. \\
\vskip\baselineskip
\vskip\baselineskip
\vskip\baselineskip
\vskip\baselineskip

\subsubsection{Resistance Measurement}
\noindent For each resistor, you will have three values of resistance: nominal (color 
code), that
measured with the ohmmeter, and that measured in the circuit.  Do all these
values agree within uncertainty for both resistors? \\
\vspace*{5mm}\\
\noindent Calculate the average resistance of each resistor using 
equation~(\ref{eq:intro:average}) and the standard deviation with 
equation~(\ref{eq:intro:standdev}).\\  
\vspace*{1.5cm}\\
\noindent Is this the value you'd report? Why?For each resistor, you will have three values of resistance: nominal, that
measured with the ohmmeter, and that measured in the circuit.  Do all these
values agree within uncertainty for both resistors? \\
\vspace*{15mm}\\
\noindent Calculate the average resistance of each resistor using 
equation~(\ref{eq:intro:average}) and the standard deviation with 
equation~(\ref{eq:intro:standdev}).\\  
\vspace*{1.5cm}\\
Is this the value you'd report? Why?

\subsubsection{Resistors in Series and Parallel}

\noindent Calculate the average and standard deviation of
these measurements. 
\vskip\baselineskip
\vskip\baselineskip
\vskip\baselineskip

\noindent Calculate the average equivalent resistance and standard 
deviation using the set of three individual resistance values from 
part~\ref{sec:DC:resist}.  
\vskip\baselineskip
\vskip\baselineskip
\vskip\baselineskip

\noindent Does the measured resistance agree with the 
calculated value, within the standard deviations of the two results?\\
\vfill

\pagebreak


\renewcommand{\thesection}{2.W}

\section{DC Circuits Worksheet Part II}


{\bf \Large Name:}~ \rule{5cm}{.1mm}~~~~~~~
{\bf \Large Day/Time:}~\rule{3cm}{.1mm}\\

\subsection{Procedure}
\subsubsection{Temperature Dependence of Resistance}

\noindent {\bf Follow the procedure in $\S$ 1.4.4 in your lab manual}.
Note that your power supply should be set to 0 Volts. {\bf You are about
to measure current.  Ammeters are connected in SERIES}.
Enter your voltage and current measurements over a widespread intervals
from 0V - 3V into 
Table~\ref{tab:DC:lightbulb}.  Remember to have the bulb lit for several
measurements and to not burn out the bulb.\\

\noindent Replace the bulb with one of the resistors from the $\S$1.W.1. 
Re-measure the resistance with the {\bf 3} ways since now
you have new resistors.  Remember that was by using the color codes,
the ohmmeter, and Ohm's Law.  Make the same measurements 
over the same range of voltage values you used for the bulb, entering
them into Table~\ref{tab:DC:resistor}.

\subsection{Computer Work}
Using the computer, make a plot of the voltage versus the current for both 
sets of measurements. Make a linear fit to the resistor graph and write
down the slope and intercept with uncertainties.  You will need {\bf 4}
graph outputs all together, 2 of the lightbulb and 2 of the resistor 
(one each of each plot).
\begin{table}[htb]
\begin{center}
\begin{tabular}{|c|c|}
\hline
\multicolumn{2}{|c|}{Resistor} \\
\hline
Slope & Intercept \\
\hline
\hspace*{5cm} & \hspace*{5cm} \\
& \\
\hline
\end{tabular}
\end{center}
\caption{Slope and Intercept for the resistor $V$ vs.\ $I$ plot.}
\end{table}

\pagebreak


\begin{table}[htb]
\begin{center}
\begin{tabular}{|c|c|}
\hline
\multicolumn{2}{|c|}{Light bulb}\\
\hline
I & V \\
\hline
\hspace*{5cm} & \hspace*{5cm} \\
& \\
\hline
& \\
& \\
\hline
& \\
& \\
\hline
& \\
& \\
\hline
& \\
& \\
\hline
& \\
& \\
\hline
& \\
& \\
\hline
& \\
& \\
\hline
& \\
& \\
\hline
& \\
& \\
\hline
\end{tabular}
\end{center}
\caption{V versus I for a light bulb.}
\label{tab:DC:lightbulb}
\end{table}



\begin{table}[t]
\begin{center}
\begin{tabular}{|c|c|}
\hline
\multicolumn{2}{|c|}{Resistor}\\
\hline
I & V \\
\hline
\hspace*{5cm} & \hspace*{5cm} \\
& \\
\hline
& \\
& \\
\hline
& \\
& \\
\hline
& \\
& \\
\hline
& \\
& \\
\hline
& \\
& \\
\hline
& \\
& \\
\hline
& \\
& \\
\hline
& \\
& \\
\hline
& \\
& \\
\hline
\end{tabular}
\end{center}
\caption{V versus I for a resistor.}
\label{tab:DC:resistor}
\end{table}


\vfill
\newpage


\subsection{Procedure}



\subsubsection{Internal Resistance of a Dry Cell} 

{\bf Follow the procedure in $\S$1.4.5 in the lab manual}.
Note that the power supply should be replaced by one of the 
batteries connected to the breadboard power contacts. 
Build the extremely simple circuit
shown in Figure~\ref{fig:DC:procintresist}. 
Enter your measurements of the voltage {\it across the battery} 
and the current
through the resistor for ten different resistors into 
Table~\ref{tab:DC:DryCell}.  Remember that you may need to put
the resistors in series and parallel to get enough variations.

\begin{table}[htb]
\begin{center}
\begin{tabular}{|c|c|}
\hline
\multicolumn{2}{|c|}{Dry cell}\\
\hline
I & V \\
\hline
\hspace*{5cm} & \hspace*{5cm} \\
& \\
\hline
& \\
& \\
\hline
& \\
& \\
\hline
& \\
& \\
\hline
& \\
& \\
\hline
& \\
& \\
\hline
& \\
& \\
\hline
& \\
& \\
\hline
& \\
& \\
\hline
& \\
& \\
\hline
\end{tabular}
\end{center}
\caption{V versus I for a dry cell.}
\label{tab:DC:DryCell}
\end{table}
\clearpage

\subsection{Computer Work}
Plot the voltage versus the current and obtain the slope and intercept of a
linear fit to the plot. 
\begin{table}[htb]
\begin{center}
\begin{tabular}{|c|c|}
\hline
\multicolumn{2}{|c|}{Dry cell} \\
\hline
Slope & Intercept \\
\hline
\hspace*{5cm} & \hspace*{5cm} \\
& \\
\hline
\end{tabular}
\end{center}
\caption{Slope and Intercept for the dry cell $V$ vs.\ $I$ plot.}
\end{table}

\subsection{Calculations \& Analysis}

\subsubsection{Temperature Dependence of Resistance}
\noindent Are the graphs linear?\\ 
\ \\
\vskip\baselineskip

\noindent What does the slope at each point of the graphs measure? (Use Ohm's law to
determine this.) \\
\ \\
\vskip\baselineskip
\vskip\baselineskip

\noindent What does the graph for the bulb measurements tell you about the temperature
dependence of resistance? \\
\ \\
\vskip\baselineskip
\vskip\baselineskip
\vskip\baselineskip
\vskip\baselineskip
\noindent Use the slope you obtained from the linear fit to calculate another value
for the resistance of the resistor. \\
\vskip\baselineskip
\vskip\baselineskip
\vspace*{2mm}
$$R=\mbox{\hspace*{3cm}}$$
\vspace*{1mm}\\
\noindent With the two resistance measurements from part~\ref{sec:DC:resist}, you now
have three experimental values for the resistance of your resistor. 
What is the average and standard deviation of the three measurements? \\
\vspace*{1.5cm} \\
\vskip\baselineskip
\vskip\baselineskip

\noindent Is the relative uncertainty in the average smaller than that in the individual 
measurements? \\
\ \\

\vskip\baselineskip
\vskip\baselineskip
\vskip\baselineskip
\vskip\baselineskip
\subsubsection{Internal Resistance of a Dry Cell}
\noindent What do the slope and $V$-intercept of the graph measure? \\
\vspace*{3mm}\\
\noindent What is the internal resistance of the battery? \\
\vspace*{2mm}Internal
$$r=\mbox{\hspace*{3cm}}$$
\vspace*{2mm}




























