\vfill
\pagebreak

%  Label worksheets by \thechapter.W
\renewcommand{\thesection}{\thechapter.W}



\section{Refraction Worksheet}

\subsection{Data \& Calculations}

Since the measurements in this section should be taken from your ray tracings, you
should attach your ray tracings to this worksheet. 

\subsubsection{Snell's Law and a Glass Slab}

At the point on your ray tracing where the incident ray meets the glass slab, measure
the angle of incidence $\theta _i,$ the angle of reflection $\theta _r,$
and the angle of refraction $\theta _t.$ Record these values (with uncertainties)
below.

\begin{center}
$\theta _i=$~ \rule{3cm}{.1mm} ~~~~
$\theta _r=$~ \rule{3cm}{.1mm}

$\theta _t=$~ \rule{3cm}{.1mm}
\end{center} 
\vspace*{.5cm}

\noindent
Assuming that the index of refraction of air, $n,$ is 1, use Snell's Law to determine
a value for the index of refraction of the plexiglass slab, $n_{slab},$ {\it with
uncertainty.} Record this value below.

\begin{center}
$n_{slab}=$~ \rule{3cm}{.1mm} 
\end{center}
{\it Sample Calculations:}
\vspace*{3cm}

\newpage
\noindent
Now measure the width, $D,$ of the plexiglass slab, and the distance $s_{meas}$
between the primary and secondary reflected rays. Record these values (with
uncertainties) below.

\begin{center}
$D=$~ \rule{3cm}{.1mm} ~~~~
$s_{meas}=$~ \rule{3cm}{.1mm}
\end{center}

\noindent
Calculate the value predicted by equation (6.2) in the lab manual for the distance
between the primary and secondary reflected rays, $s_{calc},$ {\it with uncertainty.}
Make sure you use your value of $n_{slab}$ for $n$ in this equation.
Record the calculated value below.

\begin{center}
$s_{calc}=$~ \rule{3cm}{.1mm}
\end{center}

{\it Sample Calculations:}
\newpage

\subsubsection{Total Internal Reflection}

From your ray tracing, measure the prism angle $\alpha$ and the angle of incidence of
the incident ray on the prism $\theta _i.$ Record these values (with uncertainties)
below. 
\begin{center}
$\alpha = $~ \rule{3cm}{.1mm} ~~~~
$\theta _i=$~ \rule{3cm}{.1mm}
\end{center}
\vspace*{.5cm}

\noindent
Use Snell's Law to determine the angle of refraction $\theta _r,$ {\it with
uncertainty.} (The prism has an index of refraction of $n \sim 1.52.$) 
Record your value for $\theta _r$ below.

\begin{center}
$\theta _r=$~ \rule{3cm}{.1mm}
\end{center}
{\it Sample Calculations:}
\vspace*{4cm}

\noindent
By simple geometry, you can calculate the angle of incidence of the refracted ray on
the hypotenuse of the prism, $\theta  '_i,$ by the formula

$$
\theta '_i = \alpha - \theta _r.
$$

\noindent
Calculate a value for $\theta '_i$ {\it with uncertainty.} {\bf Note:} This
instruction supersedes the instructions in the lab manual. Also, use equation (6.5) in
the lab manual to calculate the critical angle of incidence for total internal
reflection in crown glass, $\theta _{crit}.$ (Again, use $n \sim 1.52$ for the index
of refraction of the prism.) Record these values below.

\begin{center}
$\theta '_i=$~ \rule{3cm}{.1mm} ~~~~ 
$\theta _{crit}=$~ \rule{3cm}{.1mm}
\end{center} 

{\it Sample Calculations:}
\newpage


\subsection{Discussion}



\noindent
Do the sides of the plexiglass slab appear to be parallel?
\vspace*{.3cm}

\noindent
Explain how the rays that you've traced allow you to determine this.
\vspace*{1.8cm}

\noindent
Compare your value for the index of refraction of the slab, $n_{slab}$ with the
expected value of $n \sim 1.49.$ 
\vspace*{1.4cm}

\noindent 
Compare the distance you measured between the primary and secondary reflected rays,
$s_{meas}$ with the predicted value $s_{calc}.$
\vspace*{1.4cm}

\noindent
Finally, compare the value you determined for $\theta '_i$ with the value of 
$\theta _{crit}$ that you calculated.
\vspace*{1.4cm}

\noindent
Discuss the conclusions that you can draw from the results of these comparisons.
\vspace*{2.5cm}

\subsection{Conclusion}

Write a {\it brief} (that is, a one or two paragraph) conclusion for this lab (on your
own paper). In it, you should summarize the physical principles which were meant to be
illustrated in this experiment. You should also describe the degree to which your data
supported these principles.



% Go back to ordinary section numbering
\renewcommand{\thesection}{\thechapter.\arabic{section}}




