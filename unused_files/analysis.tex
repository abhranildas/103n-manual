At the moment this will just be a disjoint union of things I cleaved from my 
handout on the style of the labreport. The rest of the handout forms the 
beginnings of the Lab Report chapter. 

\section{Significant Figures}

A remark about the significant figures that you should report your 
results to is in order. To get the correct number of significant figures 
in a result:  
\begin{enumerate} 
\item Calculate the uncertainty in the quantity.
\item Round off the uncertainty to one or two digits.
\item Express the uncertainty in the same units as the quantity.
\item Round off the quantity to the last decimal place of the uncertainty.
\item Always write down the final result of a calculation with the uncertainty
and units included. 
\end{enumerate}
Use the form 
\begin{eqnarray}
(2.34 \pm 0.23) \cdot 10^3 m, \mbox{ or } 2.34 \pm 0.23 \mbox{km} \nonumber 
\end{eqnarray}
{\bf not} expressions such as 
\begin{eqnarray}
& 2.34 \cdot 10^3 m \pm 0.23 \cdot 10^3 m, \nonumber \\
& 2.34  km \pm 23 \cdot 10^1 m, \nonumber \\
& 2340 \cdot m \pm 0.23 \cdot 10^3 m. \nonumber
\end{eqnarray}

\section{Graphing}
 
In many labs you will be required to graph sets of 5 to 15 data points and make
linear fits to the data.  Such graphs should be made on graph paper of at least
4 boxes to the inch and should be large and clearly drawn.  Using a full page 
for a graph of 10 points is not unreasonable.  The scales on the axes should be
chosen appropriately,  in order that the data is well spread over the area of 
the graph. Having bunched data points leads to difficulty in reading the graph 
and loss of precision in fitting lines and calculating slopes and intercepts.  
Do not draw your axes across a full page and choose your scale in such a way 
that the data points occupy only a few cm$^2$!  Also, take care to distinguish 
between dependent and independent variables when graphing;  the quantity which 
is the function of the other in the experiment is conventionally plotted along 
the vertical axis.  If you are asked to plot $y$ vs.\ $x$,  for example,  what 
is usually meant is that $y$ depends on $x$ and should appear on the vertical 
axis.

You should be familiar with the techniques of line fitting to linear data 
points.  Note, however, that in general,  the process of fitting a line or 
curve to experimental data is theory dependent.  As this course focuses on the
experimental techniques of physics, you should avoid drawing lines or curves
through data points unless directed to.  In any case,  do not simply attempt 
to ``connect the dots'';  doing so has no physical basis and yields no insight
into  what is being examined.  Also, avoid obscuring data points when drawing
lines on a graph.  Before fitting lines you should include errorbars on two or 
three of your data points in order to take into account your experimental 
uncertainty.  The line which best fits the linear trend of the data is then 
drawn.  In addition,  you should draw the steepest and least steep lines that 
are consistent with both the trend of the data points and the errorbars.  Note 
that these lines represent overall trends in the data and need not pass 
through any specific data points. You should never just pick two convenient 
data points and draw a line through them. You want to fit the trend of the 
data, which is not represented by just a sampling of a few data points. Make 
use all of your data. The slopes and intercepts of these lines may then be 
calculated.  

When calculating the slope of a line,  select points on the line to make your 
calculation,  do not use data points, which need not fall precisely on any of 
the lines you have drawn.  Remember to follow the units through your 
calculations. The slope and intercept of the best line fit are taken to be the 
parameters of the fit to the data. Do not average the values obtained from the 
bounding lines,  these are used to calculate the uncertainty in the best fit 
values by the formulas
\begin{eqnarray}
\Delta m &=& \frac{\mid m_{steepest} - m_{least} \mid}{2} \\
\Delta b &=& \frac{\mid b_{steepest} - b_{least} \mid}{2},
\end{eqnarray}  
where $m$ and $b$ denote the slope and intercept respectively. 

As an example of proper graphing technique, consider a plot of the voltage
vs. current data:
\begin{center}
\begin{tabular}{l|l}
\multicolumn{1}{c} I(mA)  & \multicolumn{1}{c} V(V)\\
\hline
0.66 $\pm$ 0.01 & 0.6 $\pm$ 0.1 \\
1.44 & 1.8 \\
2.90 & 3.2 \\
3.90 & 3.6 \\
4.21 & 4.6 \\
5.43 & 5.3
\end{tabular}
\end{center}
\begin{figure}
\epsfxsize=14cm \epsfbox{graph.eps}
\caption{The proper way to draw a graph.}
\end{figure}
Note that errorbars are used and that the best fit line and the bounding lines
are drawn to the trend of the data, not through any data points in particular. 
The points selected to calculate slopes are not data points, they are points 
from the line\\
\begin{tabular}{ll}
best fit line: &(0.25mA, 0.3V), (5.0mA, 5.15V)\\
steepest line: &(0.5mA, 0.5V), (4.6mA, 5.0V)\\
least steep line: &(0.65mA, 1.0V), (5.2mA, 5.0V).
\end{tabular}\\
The slopes are calculated to be
\begin{eqnarray}
m &=& \frac{5.15V-0.3V}{5.0mA-0.25mA} \nonumber \\
&=& 1.02105k\Omega, \nonumber \\
m_{steepest} &=& 1.09756k\Omega , \nonumber \\
m_{least} &=& 0.87912k\Omega . \nonumber 
\end{eqnarray}
The uncertainty in the slope is calculated to no more than two digits
\begin{eqnarray}
\Delta m &=& \frac{\mid m_{steepest} - m_{least} \mid}{2} \nonumber \\
&=& \frac{|1.09756k\Omega -0.87912k\Omega |}{2} \nonumber \\
&=& 0.11k\Omega . \nonumber 
\end{eqnarray}
The slope is therefore 
\begin{eqnarray}
m = 1.02 \pm 0.11k\Omega . \nonumber
\end{eqnarray}


\section{Least Squares Analysis}
	
As an alternative to the graphical techniques above,  a linear fit to data may 
also be obtained by the least squares method.  Note, however, that you must 
still plot your data in order to visually verify the linear behavior.  The lab 
manual provides the formula for calculation of the uncertainty in the slope of 
a least squares fit as
\[ \sigma_{m} = \left[ \frac{N}{N-2} \sum_{n} \frac{d_{n}^{2}}{\Delta}
\right]^{1/2}. \]
Refer to the manual for the notation used.  It is also possible to calculate
the uncertainty in the intercept,  in the same notation we have
\[ \sigma_{b} = \left[ \frac{N}{N-2} 
\left(\sum_{n} \frac{d_{n}^{2}}{\Delta} \right) 
\left(\sum_{n} x_{i}^{2}\right) \right]^{1/2}. \]

\section{Using a Computer to Simplify Analysis}

You are strongly encouraged to make use of computer programs to simplify data
analysis.  Many computer labs on campus (such as that in the UGL) have 
computers with spreadsheet and graphing programs that can take quite a bit of 
work out of writing a lab up. However, some caveats must be made here:
\begin{enumerate}
\item You still need to provide sample calculations, though once a given 
calculation has been demonstrated, you don't have to repeat the sample 
calculation in later labs.   

\item Most computer programs will not keep track of  units for you. You must 
take extra care to be sure that your results are reported in the correct units,
especially in terms of the proper order of magnitude.

\item Similarly, most computer programs can report the result of a calculation
to a large decimal place. You need to get the proper number of significant
figures by apply the uncertainty method previously described. If you have a
program that will do this, fine. If not, you need to do this by hand. Be sure
that your final results are displayed in the format prescribed, if necessary
do this by hand as well.

\item If you use a graphing program, be sure that it can include errorbars. 
Also, be especially sure not to let the program 
``connect the dots'' on the graph. If it cannot draw a ``trend'' line, 
plot only the data points with some errorbars and draw the line fits yourself.

\item Since the axes provided by most computer graphs display only a few 
``ticks'' to indicate distance, it is hard to use such a graph to do a line 
fit by hand.  If possible use a least squares fit or use your own judgment to 
determine whether or not you need to plot your data on graph paper, by hand.
\item Least squares fitting routines are often provided by one of the following
sources: a scientific calculator, a graphing program (such as Cricket Graph 
which is on most of the UGL computers), or on a spreadsheet.  Remember that
you will need to provide uncertainties in the fit parameters and a plot
of the data.     
\end{enumerate}