\documentstyle [12pt,epsf]{article}
\pagestyle{empty}

\begin{document}
\thispagestyle{empty}

\begin{center}
{\bf\Large DC Circuits Worksheet}
\end{center}

{\bf \Large Name:}~ \rule{5cm}{.1mm}~~~~~~~
{\bf \Large Day/Time:}~\rule{3cm}{.1mm}\\

\section {Data}
\subsection {Batteries in Series}
In the table below, enter your measurements for the voltages of the
individual batteries, the batteries in series, and the batteries in opposition. Don't
forget uncertainties.

\begin{table}[h]
\begin{center}
\begin{tabular}{|c|c|}
\hline
Voltage (battery 1) & Voltage (battery 2) \\ 
\hline
\hspace*{5cm} & \hspace*{5cm}\\
& \\
\hline
\hline
Voltage in Series & Voltage in Opposition \\
\hline
& \\
& \\
\hline
\end{tabular}
\end{center}
\caption{Voltage measurements.}
\label{tab:DC:battseries}
\end{table}


\newpage
\subsection{Resistance Measurement}

\noindent
Enter your resistance measurements for each resistor in 
Table~\ref{tab:DC:resistmeas}. 

\begin{table}[htb]
\begin{center}
\begin{tabular}{|c|c|}
\hline
\multicolumn{2}{|c|}{Resistor 1} \\
\hline 
Voltage & Current  \\ 
\hline
\hspace*{3cm} & \hspace*{3cm}  \\ 
&  \\ 
\hline
Color Code & Ohmmeter  \\ 
\hline
&   \\
&   \\
\hline
\hline
\multicolumn{2}{|c|}{Resistor 2} \\
\hline 

Voltage & Current  \\
\hline
\hspace*{3cm} & \hspace*{3cm}  \\ 
&  \\ 
\hline

Color Code & Ohmmeter  \\ 
\hline
&  \\
&   \\
\hline
\end{tabular}
\end{center}
\caption{Resistance measurements.}
\label{tab:DC:resistmeas}
\end{table}

\newpage
\subsection{Resistors in Series and Parallel}

\noindent
In the table below, enter your measurements for resisors in
series and in parallel. 

\begin{table}[htb]
\begin{center}
\begin{tabular}{|c|c|}
\hline
\multicolumn{2}{|c|}{Series Resistors (Measurements)} \\
\hline 
Voltage & Current \\
\hline
\hspace*{3cm} & \hspace*{3cm}  \\ 
&  \\ 
\hline
Ohmmeter &  \\ 
\hline
&   \\
&   \\
\hline
\hline
\multicolumn{2}{|c|}{Parallel Resistors (Measurements)} \\
\hline 
Voltage & Current  \\
\hline
\hspace*{3cm} & \hspace*{3cm}  \\ 
&  \\ 
\hline
Ohmmeter &  \\
\hline
&  \\
&   \\
\hline
\hline
\end{tabular}
\end{center}
\caption{Series and parallel resistance measurements.}
\label{tab:DC:measserpar}
\end{table}

\newpage
\subsection{Temperature Dependence of Resistance}

{\bf Note:} You don't want to burn out the bulb, so don't make the voltage too
high, but be sure that the bulb is lit for several of the measurements.  \\

Now, choose a resistor.
Measure the resistance of this resistor using both the Ohmmeter and Ohm's Law.
Enter these measurements in Table~\ref{tab:DC:newres}.
Now replace the bulb in this circuit by this resistor.  
Repeat the measurements of voltage and current
over the same range of voltage values you used for the bulb, and enter these
values in Table~\ref{tab:DC:resisplot}. 


\begin{table}[htb]
\begin{center}
\begin{tabular}{|c|c|}
\hline
\multicolumn{2}{|c|}{Light bulb}\\
\hline
I & V \\
\hline
\hspace*{5cm} & \hspace*{5cm} \\
& \\
\hline
& \\
& \\
\hline
& \\
& \\
\hline
& \\
& \\
\hline
& \\
& \\
\hline
& \\
& \\
\hline
& \\
& \\
\hline
& \\
& \\
\hline
& \\
& \\
\hline
& \\
& \\
\hline
\end{tabular}
\end{center}
\caption{V versus I for a light bulb.}
\label{tab:DC:ltbulbplot}
\end{table}

\begin{table}[htb]
\begin{center}
\begin{tabular}{|c|c|c|}
\hline
\multicolumn{3}{|c|}{Resistor} \\
\hline 
Voltage & Current & Ohm's Law Resistance \\ 
\hline
\hspace*{3cm} & \hspace*{3cm} & \hspace*{3cm} \\ 
& &  \\ 
\hline
Ohmmeter &  &  \\ 
\hline
& &  \\
& &  \\
\hline
\end{tabular}
\end{center}
\caption{Resistance Measurements.}
\label{tab:DC:newres}
\end{table}

\begin{table}[htb]
\begin{center}
\begin{tabular}{|c|c|}
\hline
\multicolumn{2}{|c|}{Resistor}\\
\hline
I & V \\
\hline
\hspace*{5cm} & \hspace*{5cm} \\
& \\
\hline
& \\
& \\
\hline
& \\
& \\
\hline
& \\
& \\
\hline
& \\
& \\
\hline
& \\
& \\
\hline
& \\
& \\
\hline
& \\
& \\
\hline
& \\
& \\
\hline
& \\
& \\
\hline
\end{tabular}
\end{center}
\caption{V versus I for a resistor.}
\label{tab:DC:resisplot}
\end{table}


\subsection{Internal Resistance of a Dry Cell}
\noindent
Enter your voltage versus current measurements for the dry cell in
Table~\ref{tab:DC:intres}.

\newpage
\begin{table}[htb]
\begin{center}
\begin{tabular}{|c|c|}
\hline
\multicolumn{2}{|c|}{Dry cell}\\
\hline
I & V \\
\hline
\hspace*{5cm} & \hspace*{5cm} \\
& \\
\hline
& \\
& \\
\hline
& \\
& \\
\hline
& \\
& \\
\hline
& \\
& \\
\hline
& \\
& \\
\hline
& \\
& \\
\hline
& \\
& \\
\hline
& \\
& \\
\hline
& \\
& \\
\hline
\end{tabular}
\end{center}
\caption{V versus I for a dry cell.}
\label{tab:DC:intres}
\end{table}

\newpage
\section {Data Analysis and Calculations}

\subsection{Batteries in Series}
\noindent
From the battery voltages you measured,
calculate the voltages you expect to measure for the batteries in series and in
opposition, and enter these below. Be sure to present
sample calculations (including uncertainty calculations) in the space provided.

\begin{center}
Expected Voltage in Series:\\
Expected Voltage in Opposition:
\end{center}

\noindent
{\it Sample Calculations:}


\subsection{Resistance Measurement}

\noindent
From your voltage and current measurements in Table~\ref{tab:DC:resistmeas},
use Ohm's Law to determine the resistances of each resistor with uncertainties.
(Show sample calculations in space indicated.) Enter these values below:

\begin{center}
Ohm's Law resistance for Resistor 1:\\
Ohm's Law resistance for Resistor 2:
\end{center}
 
\noindent
Using the above values, and the ohmmeter and color code values, calculate the 
average resistance of each resistor using 
equation (0.1) in the lab manual and the standard deviation with 
equation (0.2) in the lab manual.  Enter these values below.

\begin{center}
Average Resistance for Resistor 1:\\
Average Resistance for Resistor 2:
\end{center}

\noindent
{\it Sample Calculations:} 



\newpage
\subsection{Resistors in Series and Parallel}

\noindent
From your voltage and current measurements in Table~\ref{tab:DC:measserpar},
use Ohm's Law to determine the equivalent resistances of the series and
parallel configurations of resistors with uncertainties. Enter these values below. 

\begin{center}
Ohm's Law resistance for Resistors in Series:\\
Ohm's Law resistance for Resistors in Parallel:
\end{center}
 
\noindent
Calculate the average and standard deviation of your two measurements for each
circuit, and enter these values below.

\begin{center}
Average Resistance for Resistors in Series:\\
Average Resistance for Resistors in Parallel:
\end{center} 

\noindent
Now, from your resistance measurements of the {\it individual} resistors, 
(i.e. meaurements in Table~\ref{tab:DC:resistmeas}),
calculate the {\it expected} equivalent resistances of the resistors in 
series and parallel. You should calculate a new value for each type of
measurement.  
Then calculate the average and standard deviation of the expected values.
Enter all of these values in Table~\ref{tab:DC:calcserpar}. 

\begin{table}[htb]
\begin{center}
\begin{tabular}{|c|c|c|}
\hline
\multicolumn{3}{|c|}{Series Resistors (Calculated)} \\
\hline 
Color Codes & Ohmmeter & Ohm's Law Resistance \\
\hline
\hspace*{3cm} & \hspace*{3cm} & \hspace*{3cm} \\ 
& &  \\ 
\hline
Average &  & \\ 
\hline
& &  \\
& &  \\
\hline
\hline
\multicolumn{3}{|c|}{Parallel Resistors (Calculated)} \\
\hline 
Color Codes & Ohmmeter & Ohm's Law Resistance \\
\hline
\hspace*{3cm} & \hspace*{3cm} & \hspace*{3cm} \\ 
& &  \\ 
\hline
Average &  & \\
\hline
& &  \\
& &  \\
\hline
\hline
\end{tabular}
\end{center}
\caption{Series and parallel resistance measurements.}
\label{tab:DC:calcserpar}
\end{table}

\noindent
{\it Sample Calculations:}


\newpage

\subsection{Temperature Dependence of Resistance}

Using the computer, make a plot of the voltage versus the current for both 
sets of measurements. Make a linear fit to the resistor graph and write
down the slope and intercept in Table~\ref{tab:DC:slopeinter}.

\begin{table}[htb]
\begin{center}
\begin{tabular}{|c|c|}
\hline
\multicolumn{2}{|c|}{Resistor} \\
\hline
Slope & Intercept \\
\hline
\hspace*{5cm} & \hspace*{5cm} \\
& \\
\hline
\end{tabular}
\end{center}
\caption{Slope and Intercept for the resistor $V$ vs.\ $I$ plot.}
\label{tab:DC:slopeinter}
\end{table}

\subsection{Internal Resistance of a Dry Cell}
Plot the voltage versus the current and obtain the slope and intercept of a
linear fit to the plot.  Enter these values in Table~\ref{tab:DC:battslope}.

\begin{table}[htb]
\begin{center}
\begin{tabular}{|c|c|}
\hline
\multicolumn{2}{|c|}{Dry cell} \\
\hline
Slope & Intercept \\
\hline
\hspace*{5cm} & \hspace*{5cm} \\
& \\
\hline
\end{tabular}
\end{center}
\caption{Slope and Intercept for the dry cell $V$ vs.\ $I$ plot.}
\label{tab:DC:battslope}
\end{table}

\newpage

\section{Discussion}
\subsection{Batteries in Series}
Compare your results with those that you would expect from your knowledge of 
voltage sources in series.  Do the values match within uncertainty? \\

\subsection{Resistance Measurement}
For each resistor, you will have four values of resistance: nominal, that
measured with the ohmmeter, that measured in the circuit, and the average. 
Do all these values agree within uncertainty for both resistors? 
  
\vspace*{1cm}
\noindent
Which of these values would you report as {\it the} resistance? Why?
\indent


\subsection{Resistors in Series and Parallel}
From the results for each circuit, answer the following question, and discuss
the conclusions which can be drawn from each of your answers. 
Does the measured resistance agree with the 
calculated equivalent resistance, within the standard deviations of the 
two results? 

\subsection{Temperature Dependence of Resistance}

\noindent
Are either of the graphs linear?\\ 
\ \\
\noindent
What does the slope at each point of the graphs measure? (Use Ohm's law to
determine this.) 
\vspace*{2cm}

\noindent
What does the graph for the bulb measurements tell you about the temperature
dependence of resistance? 
\vspace*{2cm}

\noindent
Use the slope you obtained from the linear fit to calculate another value
for the resistance of the resistor. \\
\vspace*{2mm}
$$R=\mbox{\hspace*{3cm}}$$
\vspace*{1mm}\\
With the two other measurements, you now
have three experimental values for the resistance of your resistor. 
What is the average and standard deviation of the three measurements? \\
\vspace*{1.5cm} \\
Compare the relative uncertainty in the average to those in the individual 
measurements. \\

\subsection{Internal Resistance of a Dry Cell}

What do the slope and $V$-intercept of the graph measure? \\
\vspace*{1cm}\\
What is the internal resistance of the battery? \\
\vspace*{2mm}
$$r=\mbox{\hspace*{3cm}}$$


\section{Conclusion}
Write a {\it brief} conclusion for this lab (on your own paper). This conclusion should
contain a summary of what physical principles were supposed to be demonstrated in
these experiment and a {\it brief} description of the degree to which your data 
supported these principles.


\end{document}




