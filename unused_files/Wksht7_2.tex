\vfill
\pagebreak

%  Label worksheets by \thechapter.W
\renewcommand{\thesection}{\thechapter.W}

\section{Diffraction and Interference of Light Worksheet}
\subsection{Data}

\subsubsection{Single Slit Diffraction}

(Follow procedure in Section 7.4.1 of Lab Manual (pp 136-7))
\vspace*{.5cm}

\noindent
Measure distance from slit to screen, $L$ and record this value below with uncertainty.

\begin{center}
$L=$~ \rule{3cm}{.1mm} 
\end{center}
\vspace*{.5cm}

\noindent
Make sure you obtain {\bf two} traces, one for a relatively large slit
width and the other for a relatively small slit width. Label these
traces as you take them.

\subsubsection{Multiple Slit Diffraction/Interference Patterns}

(Follow procedure in Section 7.4.2 of Lab Manual (pp 137-8))

\noindent
Make sure you trace the {\bf two slit} interference pattern, and make
sure you are able to trace at least {\bf eight primary} maxima. (Four
primary maxima on each side of the central maximum is preferable).

\noindent 
Record the slit width $a$ and the slit spacing $d$ below.

\begin{center}
$a=$~ \rule{3cm}{.1mm}~~~~ $d=$~ \rule{3cm}{.1mm}
\end{center}
\vspace*{.5cm}

\noindent
Record below the results of your observations of the {\it qualitative}
properties of the interference patterns resulting from 3, 4, and 5
slits. In particular, answer the following two questions.
\vspace*{.5cm}

\noindent 
For each of the 3, 4, and 5 slit patterns, how many secondary maxima
appear between primary maxima?
\vspace*{.5cm}

\noindent
What happens to the size of the primary maxima as you increase the
number of slits?
\vspace*{.5cm}

\subsubsection{Diffraction Grating Pattern}

(Follow procedure in Section 7.4.3 of Lab Manual (pp 138-9))

\noindent
Measure the grating to screen distance, $L_g.$ Record this with
uncertainty below.  Also, record the number of lines/cm for the
diffraction grating, $s,$ below.

\begin{center}
$L_g=$~ \rule{3cm}{.1mm} ~~~~ $s=$~ \rule{3cm}{.1mm}
\end{center}
\vspace*{.5cm}

\noindent
Make sure you are able to trace {\bf five} dots from the grating's
interference pattern.

\newpage

\subsection{Data Analysis and Calculations}

\subsubsection{Multiple Slit Diffraction/Interference Patterns}

Determine the orders of each of the primary maxima on your trace of
the two-slit interference pattern. Refer to Figure 7.9 on page 138 of
the Lab Manual to see how.
\vspace*{.5cm}

\noindent
Measure the distances $y_n$ from the $n$th primary maximum to the
center of the central maximum for at least eight primary maxima. {\bf
Note:} The values of $y_n$ for negative $n$ should be taken to be
negative. Record your eight $y_n$ vs. $n$ measurements in the table
below.

\begin{table}[htb]
\begin{center}
\begin{tabular}{|c|c|c|c|c|}
\hline
\multicolumn{5}{|c|}{Measurements of $y_n$ vs. $n$ for Two-slit Diffraction Pattern.} \\
\hline
Distance ($y_n$) & Order ($n$) & & Distance ($y_n$) & Order ($n$) \\
\hline
\hspace*{3cm} & \hspace*{3cm} & \hspace*{.3cm} & \hspace*{3cm} & \hspace*{3cm} \\
& & & & \\
\hline
& & & & \\ & & & & \\
\hline
& & & & \\ & & & & \\
\hline
& & & & \\ & & & & \\
\hline
\end{tabular}
\end{center}
\caption{$y_n$ vs. $n$ measurements for two-slit diffraction pattern.}
\label {tab:DI:twoslit}
\end{table}

\noindent
Make a plot (with error bars) of $y_nd/L$ vs. $n$ with error
bars. Find the slope of the weighted best fit line, and record this
below with uncertainty.

\begin{center}
Slope1=~ \rule{3cm}{.1mm}
\end{center}
\vspace*{.5cm}
\noindent
{\it Sample Calculations:}
\newpage

\subsubsection{Diffraction Grating Pattern}

On your trace of the diffraction grating interference pattern, assign
a value for $n$ to each of the dots. Refer to figure 7.10 on page 138
of the Lab Manual to see how.
\vspace*{.5cm}

\noindent
Measure the distances $y_n$ from the $n$th dot to the center dot for
all five of the dots. {\bf Note:} Again, the values of $y_n$ for
negative $n$ should be taken to be negative. Record your five $y_n$
vs. $n$ measurements in the table below.

\begin{table}[htb]
\begin{center}
\begin{tabular}{|c|c|}
\hline
\multicolumn{2}{|c|}{Measurements of $y_n$ vs. $n$} \\
\hline
Distance ($y_n$) & Order ($n$) \\
\hline
\hspace*{3cm} & \hspace*{3cm}  \\
& \\
\hline
& \\
&  \\
\hline
& \\
& \\
\hline
&  \\
&  \\
\hline
&  \\
&  \\
\hline
\end{tabular}
\end{center}
\caption{$y_n$ vs. $n$ measurements for diffraction grating pattern.}
\label {tab:DI:Grating}
\end{table}

\noindent
Now, plot $y_nd/ \sqrt{y_n^2 + L^2}$ vs. $n$ with error bars. Find the slope of the
weighted best fit line, and record this below with uncertainty.

\begin{center}
Slope2=~ \rule{3cm}{.1mm}
\end{center}
\vspace*{.5cm}
\noindent
{\it Sample Caluclations:}

\newpage
\noindent
From your value for Slope1, determine the value of the wavelength of the laser light
for your two-slit interference pattern, $\lambda _1,$ with uncertainty. From your
value of Slope2, determine the value of the wavelength of the laser light for your
diffraction grating interference pattern, $\lambda _2$ with uncertainty. Record these
values below.

\begin{center}
$\lambda _1=$~ \rule{3cm}{.1mm} ~~~~
$\lambda _2=$~ \rule{3cm}{.1mm}
\end{center}
\vspace*{.5cm}

\noindent
{\it Sample Calculations:}
\vspace*{1cm}

\noindent
From $\lambda _1$ and $\lambda _2$ calculate the average wavelength with standard
deviation. {\bf Note:} Use equations (0.1) and (0.2) on page 5 of the Lab Manual for
this. Record the average value $\lambda _{avg}$ below with uncertainty.

\begin{center}
$\lambda _{avg}=$~ \rule{3cm}{.1mm} 
\end{center}
\vspace*{.5cm}

\noindent
{\it Sample Calculations:}

\newpage

\subsection{Discussion}
\subsubsection{Single Slit Diffraction}
Refer to the traces you made of the single-slit diffraction patterns.
Is the small angle approximation valid for analyzing these traces? Explain.
\vspace*{1cm}

\noindent 
What happened to the width of the spots in the single-slit diffraction pattern as you
increased the slit width?
\vspace*{.5cm}

\noindent
What happened to the separation between the spots in the single-slit diffraction
pattern as you increased the slit-width?
\vspace*{.5cm}

\noindent
Do your answers to the above two questions agree qualitatively with the results
obtained using the small-angle approximation in Section 7.2.6? Explain by citing the
relevant results.
\vspace*{2cm}

\subsubsection{Multiple Slit Diffraction/Interference Patterns}
What is responsible for the existence of missing orders in a multiple slit
interference pattern? (Hint: examine Figure 7.8 on page 135 of the Lab Manual 
carefully.)
\vspace*{1.4cm}

\noindent
Is the small-angle approximation valid for analyzing your trace of the two-slit
intereference pattern? Explain by comparing your largest $y_n$ value to $L.$
\vspace*{1cm}

\noindent
Is your plot of $y_nd/L$ vs. $n$ linear?
\vspace*{.3cm}

\noindent
Compare your value of $\lambda _1$ to the known wavelength of He-Ne light, 632.8 nm.
\newpage

\noindent
Does your observation of how the sizes of the primary maxima behave as you increase
the number of slits agree or disagree with the discussion in section 7.2.7 of the Lab
Manual (pp 134-5)? Explain.
\vspace*{2cm}
  
\noindent
Discuss what conclusions can be drawn from your answers to the above three questions.
\vspace*{3cm}

\noindent
Referring to your trace of the two-slit interference pattern and the qualitative
observations you made about the 3, 4, and 5 slit interference patterns, write an
equation that relates the number of slits, $N$, to the number of secondary maxima, 
$M,$ that occur in the corresponding interference pattern.
\newpage

\subsubsection{Diffraction Grating Pattern}
Is your plot of $y_nd/ \sqrt{y_n^2 + L^2}$ linear?
\vspace*{.3cm}

\noindent
Compare your value of $\lambda _2$ to the known wavelength of He-Ne light, 632.8 nm.
\vspace*{1.4 cm}

\noindent
Discuss what conclusions can be drawn from your answers to the above two questions.
\vspace*{3cm}

\noindent
Compare your value for $\lambda _{avg}$ with the known wavelength.
\vspace*{1.4cm}

\noindent
Which of the two methods of measuring wavelength is the most {\it accurate}?
\vspace*{.5cm}

\noindent
Is this expected? Why?
\vspace*{1cm}

\noindent
Do you gain anything by having made two separate measurements? What?
\vspace*{1.4cm} 


\subsection{Conclusion}

Write a {\it brief} (that is, a one or two paragraph) conclusion for this lab (on your
own paper). In it, you should summarize the physical principles which were meant to be
illustrated in this experiment. You should also describe the degree to which your data
supported these principles.




