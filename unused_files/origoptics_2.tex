\section{Introduction}

Now that you have studied how matter bends light, we can use this effect 
to form images of various objects. {\bf We will want to tie the discussion of 
this lab in to the refraction lab, i.e.\ use what the students ``know'' 
already.} We will do this using lenses in this lab, but you can also use 
mirrors. This simple technology forms the basis of an enormous part of our 
economy. It is also a lot of fun to do.

We will make the approximation that our lenses are thin. This means that 
the distances from the object to the lens and from the lens to the image
are much larger than the thickness of the lens itself. This is not a big
restriction, because most lenses we use usually satisfy this simple 
criterion. You should check Serway [2], Chapter 36 and verify that this is the 
correct criterion.  You might continue reading the chapter after you've done 
this. This approximation simplifies the mathematical description of 
imaging optics; it boils down to one equation, the {\em thin lens equation}:
\begin{equation}
\frac{1}{o} + \frac{1}{i} = \frac{1}{f}, \label{eq:image}
\end{equation}
where $o$ is the object distance, $i$ is the image distance, and $f$ is 
the focal length of the lens (see Figure \ref{fig:lens1}). {\bf We need to 
define the focal length with a figure.}
\begin{figure}[hb]
\hspace*{2.5cm} \epsfxsize=8cm \epsfysize=4cm \epsfbox{optics/lens1.eps}
\caption{Lens with focal length $f$ and corresponding object $o$ and image
$i$ distances.}
\label{fig:lens1}   
\end{figure}
The image and object distances 
are easy to measure, while the focal length of the lens typically 
appears somewhere on the lens itself. The radii of curvature of the two 
surfaces of lens and the index of refraction of the glass completely
determine the focal length of a thin lens via
$$
\frac{1}{f} = ( n - 1 ) \left[\frac{1}{r_1} - \frac{1}{r_2} \right],
$$
called the {\em lens maker's equation}. In this lab, we will only use the 
thin lens equation, equation (\ref{eq:image}).

To use equation (\ref{eq:image}) correctly, we must examine the sign 
convention that accompanies it. This arises from the possibility of negative
distances in equation (\ref{eq:image}). The sign convention goes as follows:
{\em positive} distances are associated with {\em real} images; {\em 
negative} distances are associated with {\em virtual} images. This 
is extremely important, for real images can be projected onto a screen, and
virtual images can be seen by looking into the lens.

Another physical quantity associated with lenses is their magnifying or
reducing capability. The {\em lateral magnification}, $m$, of a lens is 
\begin{equation}
m = - \frac{i}{o}. \label{eq:mag}
\end{equation}
{\em Positive} magnification indicates an {\em upright} image, while
{\em negative} magnification indicates an {\em inverted} image. Thus, there
are four possible image types that a thin lens can produce: real upright,
real inverted, virtual upright, and virtual inverted.

\section{Procedure}

We will do four experiments involving thin lenses. The basic set up for
these experiments appears in Figure 2. The light source, mounted with the 
slide of a small arrow, should be placed at the end of a meter stick, with 
the lens or lenses placed on the meter stick. This will form an image down 
the meter stick which you will find using a 3$\times$5 notecard. This 
allows you to measure the image and object distances for various lens 
configurations. Make sure you record your uncertainties.  The best way to do 
this is to record the position of one of the edges of the optical mount and 
then add the (constant) distance from the edge of the mount to the lens.

{\bf Part 1:} Place the 238 mm lens on the meter stick an arbitrary
distance from the light source. Find the image and record the image  and
object distances. Now, move the lens  to another position and do the same
thing. Repeat this process until you have four image-object distance pairs.
Use the thin lens equation (equation \ref{eq:image}) to determine the focal
length of the lens. Specifically, plot $1/i$ versus $1/o$ with two errorbars,
is this linear? Find the $1/i$-intercept of the graph (with uncertainty) and 
use the thin lens equation to determine $f$ from it (with uncertainty.) 
{\bf This includes the use of a graph to find $f$.}
Compare this value of $f$ with the value given on the lens.

{\bf Part 2:} Repeat the previous experiment with the other lens (the 
136 mm lens). Make sure to include uncertainties and that you calculate $f$. 
Would you believe the nominal focal length of some other lens, based on your 
results from these two experiments?

{\bf Part 3:} Using one of the lenses, put it on the meter stick at an 
arbitrary distance and find the associated image. Record the image and 
object distances and also record the size of the image and the size of the 
object with uncertainties. Note whether the image is inverted or upright. 
Using equation (\ref{eq:mag}),determine the lateral magnification. Now, 
multiply the object size by the lateral magnification. Compare this to the 
size of the image. Repeat this process for the other lens. Propagate all your 
uncertainties!

{\bf Part 4:} Build the configuration shown in Figure \ref{fig:lens2}. 
\begin{figure}
\hspace*{7mm} \epsfxsize=12.5cm  \epsfbox{optics/lens2.eps}
\caption{Part 4 configuration.}
\label{fig:lens2}
\end{figure}Place 
the 136 mm lens at {\em half its \underline{measured} focal length} away from 
the object. Determine the position of the image; is
it real or virtual? Now, place the 238 mm lens at a distance of {\em twice
its \underline{measured} focal length from the position of the \underline
{image} of the first lens}. Predict where the image of the second lens should
be. Using this as a guide, locate the image of the compound system and record 
your measurement with uncertainty. How close is the prediction? Measure the 
size of the final image and compute the magnification of the compound lens 
system. Be sure to record your measurement with uncertainty. How does it 
relate to the magnification of the two individual lenses in this configuration?
Is this what you would expect?  How close is the prediction? Measure the size 
of the final image and compute the magnification of the compound lens system. 
How does it relate to the magnification of the two individual lenses in this 
configuration? Is this what you would expect? To get a feel for what is going 
on, draw a ray diagram to represent what happens in the compound system. 
(See Haliday, Resnick and Walker [1], or Serway [2].)

\section{References}

[1] Haliday, David, Robert Resnick and Jearl Walker, {\em Fundamentals of 
Physics, 4th Edition}, John Wiley and Sons, 1993.

\noindent [2] Serway, Raymond A., {\em Physics for Scientists and Engineers, 
3rd Edition}, Saunders College, 1990.  